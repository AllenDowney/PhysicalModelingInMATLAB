\chapter{Zero-finding}

In this chapter we'll use the MATLAB function {\tt fzero} to find roots of nonlinear equations.


\section{Nonlinear equations}

\index{nonlinear equation}
\index{equation!nonlinear}

What does it mean to ``solve'' an equation?  That may seem like an
obvious question, but I want to take a minute to think about it,
starting with a simple example: let's say that we want to know the
value of a variable, $x$, but all we know about it is the relationship
$x^2 = a$.

If you have taken algebra, you probably know how to ``solve'' this
equation: you take the square root of both sides and get
$x = \pm \sqrt{a}$.  Then, with the satisfaction of a job well done,
you move on to the next problem.

\index{square root}

But what have you really done?  The relationship you derived is
equivalent to the relationship you started with---they contain the
same information about $x$---so why is the second one preferable
to the first?

There are two reasons.  One is that the relationship is now {\bf explicit}
in $x$: because $x$ is all alone on the left side, we can treat
the right side as a recipe for computing $x$, assuming that we
know the value of $a$.

\index{explicit computation}
\index{computation!explicit}

The other reason is that the recipe is written in terms of operations
we know how to perform.  Assuming that we know how to compute square
roots, we can compute the value of $x$ for any value of $a$.

When people talk about solving an equation, what they usually mean
is something like ``finding an equivalent relationship that is
explicit in one of the variables''.  In the context of this book,
that's what I will call an {\bf analytic solution}, to distinguish
it from a {\bf numerical solution}, which is what we are going to
do next.

\index{analytic solution}
\index{numerical solution}

To demonstrate a numerical solution, consider the equation $x^2 - 2x =
3$.  You could solve this analytically, either by factoring it or by
using the quadratic equation, and you would discover that there are
two solutions, $x=3$ and $x=-1$.  Alternatively, you could solve it
numerically by rewriting it as $x = \pm \sqrt{2x+3}$.

This equation is not explicit, since $x$ appears on both sides, so
it is not clear that this move did any good at all.  But suppose
that we had some reason to expect there to be a solution near 4.
We could start with $x=4$ as an ``initial guess,'' and then use
the equation $x = \sqrt{2x+3}$ iteratively to compute successive
approximations of the solution.\footnote{To understand why this
works, see \url{https://en.wikipedia.org/wiki/Fixed-point_iteration}.}

\index{fixed-point iteration}

Here's what happens:

\begin{code}
>> x = 4;
>> x = sqrt(2*x+3)
x = 3.3166

>> x = sqrt(2*x+3)
x = 3.1037

>> x = sqrt(2*x+3)
x = 3.0344

>> x = sqrt(2*x+3)
x = 3.0114

>> x = sqrt(2*x+3)
x = 3.0038
\end{code}

After each iteration, {\tt x} is closer to the correct answer,
and after 5 iterations, the relative error is about 0.1\%, which
is good enough for most purposes.

\index{numerical method}

Techniques that generate numerical solutions are called
{\bf numerical methods}.  
The nice thing about the method I
just demonstrated is that it is simple, but it doesn't always
work, and it is not often used in practice.
We'll see better alternatives soon.



\section{Zero-finding}
\label{zero}

A nonlinear equation like $x^2 - 2x = 3$ is a statement of
equality that is true for some values of $x$ and false for
others.  A value that makes it true is a solution;
any other value is a non-solution.  But for any given non-solution,
there is no sense of whether it is close or far from a solution,
or where we might look to find one.

\index{zero-finding}

To address this limitation, it is useful to
rewrite non-linear equations as zero-finding problems:

\begin{itemize}

\item The first step is to define an ``error function'' that computes how far
a given value of $x$ is from being a solution.

In this example, the error function is

\begin{equation}
f(x) = x^2 - 2x -3
\end{equation}

Any value of $x$ that makes $f(x) = 0$ is also a solution
of the original equation.

\index{error function}
\index{function!error}

\item The next step is to find values of $x$ that make
$f(x) = 0$.  These values are called zeros of the
function, also called {\bf roots}.

\index{root}

\end{itemize}

Zero-finding lends itself to numerical solution because we can
use the values of $f$, evaluated at various values of $x$, to
make reasonable inferences about where to look for zeros.

And one of the best numerical methods is available as a built-in MATLAB function, {\tt fzero}.


\section{{\tt fzero}}
\label{fzero}

\index{fzero@{\tt fzero}}

In order to use {\tt fzero}, you have to define a MATLAB function
that computes the error function you derived from the original
nonlinear equation, and you have to provide an initial guess at
the location of a zero.

We've already seen an example of an error function:

\begin{code}
function res = error_func(x)
    res = x^2 - 2*x -3;
end
\end{code}

You can call {\tt error\_func} from the {\sf Command Window}, and
confirm that there are zeros at 3 and -1.

\begin{code}
>> error_func(3)
ans = 0

>> error_func(-1)
ans = 0
\end{code}

But let's pretend that we don't know where
the roots are; we only know that one of them is near 4.  Then
we could call {\tt fzero} like this:

\begin{code}
>> fzero(@error_func, 4)
ans = 3.0000
\end{code}

Success!  We found one of the zeros.

The first argument is a
{\bf function handle} that names the M-file that evaluates
the error function.  The {\tt @} symbol allows us to name the
function without calling it.  The interesting thing here is
that you are not actually calling {\tt error\_func} directly;
you are just telling {\tt fzero} where it is.  In turn, {\tt fzero}
calls your error function --- more than once, in fact.

\index{function handle}
\index{handle!function}

The second argument is the initial guess.  If we provide a different
initial guess, we get a different root (at least sometimes).

\begin{code}
>> fzero(@error_func, -2)
ans = -1
\end{code}

Alternatively, if you know two values that bracket the root,
you can provide both:

\begin{code}
>> fzero(@error_func, [2,4])
ans = 3
\end{code}

The second argument is a vector that contains two elements.  

\index{vector}

You might be curious to know how many times {\tt fzero} calls your
function, and where.  If you modify {\tt error\_func} so that it displays
the value of {\tt x} every time it is called and then run {\tt fzero}
again, you get:

\begin{code}
>> fzero(@error_func, [2,4])
x = 2
x = 4
x = 2.75000000000000
x = 3.03708133971292
x = 2.99755211623500
x = 2.99997750209270
x = 3.00000000025200
x = 3.00000000000000
x = 3
x = 3
ans = 3
\end{code}

Not surprisingly, it starts by computing $f(2)$ and $f(4)$.  Then it computes a point in the interval, $2.75$ and evaluates $f$ there.  After each iteration, the interval gets smaller and the guess gets closer to the true root.
{\tt fzero} stops when the interval is so small that the estimated
zero is correct to about 15 digits.
 
If you would like to know more about how {\tt fzero} works, see Section~\ref{howfzero}.


\section{What could go wrong?}

The most common problem people have with {\tt fzero} is leaving
out the {\tt @}.  In that case, you get something like:

\begin{code}
>> fzero(error_func, [2,4])
Not enough input arguments.

Error in error_func (line 2)
    res = x^2 - 2*x -3;
\end{code}

The error occurs because MATLAB treats the first argument as a function call, so it calls {\tt error\_func} with no arguments.

\index{output variable}
\index{variable!output}

Another common problem is writing an error function that never
assigns a value to the output variable.  In general, functions should
{\em always} assign a value to the output variable, but MATLAB doesn't
enforce this rule, so it is easy to forget.  For example, if you
write:

\begin{code}
function res = error_func(x)
    y = x^2 - 2*x -3
end
\end{code}

and then call it from the {\sf Command Window}:

\begin{code}
>> error_func(4)
y = 5
\end{code}

It looks like it worked, but don't be fooled.  This function assigns
a value to {\tt y}, and it displays the result, but when the function
ends, {\tt y} disappears along with the function's workspace.
If you try to use it with {\tt fzero}, you get

\begin{code}
>> fzero(@error_func, [2,4])
y = -3

Error using fzero (line 231)
FZERO cannot continue because user-supplied function_handle ==>
error_func failed with the error below.

Output argument "res" (and maybe others) not assigned during call
to "error_func".
\end{code}

If you read it carefully, this is a pretty good error message,
provided you understand that ``output argument'' and ``output variable'' are the same thing.

\index{argument}
\index{output argument}

You would have seen the same error message when you called {\tt
error\_func} from the interpreter, if you had assigned the result
to a variable:

\begin{code}
>> x = error_func(4)
y = 5

Output argument "res" (and maybe others) not assigned during
call to "error_func".
\end{code}

You can avoid all of this if you remember these two rules:

\begin{itemize}

\item Functions should assign values to their output
variables.\footnote{Well, ok, there are exceptions, including {\tt
find\_triples}. Functions that don't return a value are sometimes
called ``commands'', because they do something (like display
values or generate a figure) but either don't have an output
variable or don't make an assignment to it.}

\item When you call a function, you should do something with
the result (either assign it to a variable or use it as part of an
expression, etc.).

\end{itemize}

When you write your own functions and use them yourself, it is easy
for mistakes to go undetected.  But when you use your functions with
MATLAB functions like {\tt fzero}, you have to get it right!

Yet another thing that can go wrong: if you provide an interval for the
initial guess and it doesn't actually contain a root, you get

\begin{code}
>> fzero(@error_func, [0,1])
Error using fzero (line 272)
The function values at the interval endpoints must differ in sign.
\end{code}

\index{interval}

There is one other thing that can go wrong when you use {\tt fzero}, but
this one is less likely to be your fault.  It is possible that {\tt fzero}
won't be able to find a root.

{\tt fzero} is generally pretty robust, so you may never have a problem, but you should remember that there is no guarantee that {\tt fzero} will work, especially if you provide a single value as an initial guess.  Even if you provide an interval that brackets a root, things can still go wrong if the error function is discontinuous.


\section{Choosing an initial guess}

The better your initial guess (or interval) is, the more likely
it is that {\tt fzero} will work, and the fewer iterations it will
need.

When you are solving problems in the real world, you will usually
have some intuition about the answer.  This intuition is often enough
to provide a good initial guess.

\index{ezplot@{\tt ezplot}}
\index{plot!{\tt ezplot}}

Another approach is to plot the function and see if you can
approximate the zeros visually.  If you have a function, like
{\tt error\_func} that takes a scalar input variable and returns
a scalar output variable, you can plot it with {\tt ezplot}:

\begin{code}
>> ezplot(@error_func, [-2,5])
\end{code}

\index{ezplot}

The first argument is a function handle; the second is the interval you want to plot the function in.

By examining the plot, you can estimate the location of the two roots.


\section{Vectorizing functions}

\index{vectorizing}
\index{function!vectorizing}

With this example, you might get the following warning\footnote{In Octave it's an error, so you have to vectorize the function.}:

\begin{code}
Warning: Function failed to evaluate on array inputs;
vectorizing the function may speed up its evaluation and 
avoid the need to loop over array elements. 
\end{code}

This means that MATLAB tried to call \verb"error_func" with a vector, and it failed. 
The problem is that it uses \verb"*" and \verb"^" operators; as we saw in Section~\ref{elementwise}, those operators don't do what we want, which is {\em elementwise} multiplication and exponentiation.

\index{elementwise operator}
\index{operator!elementwise}

If you rewrite \verb"error_func" like this:

\begin{code}
function res = error_func(x)
    res = x.^2 - 2.*x -3;
end
\end{code}

The warning message goes away, and {\tt ezplot} runs faster, for what it's worth.



\section{More name collisions}

Functions and variables occupy the same workspace, which means
that whenever a name appears in an expression, MATLAB starts by looking
for a variable with that name, and if there isn't one, it looks for
a function.

\index{name collision}
\index{collision!name}
\index{shadow}

As a result, if you have a variable with the same name as a function,
the variable {\bf shadows} the function.  For example, if you assign
a value to {\tt sin}, and then try to use the {\tt sin} function, you
{\em might} get an error:

\begin{code}
>> sin = 3;
>> x = 5;
>> sin(x)
Index exceeds the number of array elements (1).

'sin' appears to be both a function and a variable.
If this is unintentional, use 'clear sin' to remove 
the variable 'sin' from the workspace.
\end{code}

Since the value we assigned to {\tt sin} is a scalar, and a scalar is really a 1x1 matrix, MATLAB tries to access the 5th element of the matrix and finds that there isn't one.

In this case MATLAB is able to detect the error, and the error message is pretty helpful.
But if the value of {\tt sin} was a vector, or if the value of {\tt x} was smaller, you would be in trouble.  For example:

\begin{code}
>> sin = 3;
>> sin(1)
ans = 3
\end{code}

Just to review, the sine of 1 is not 3!

You can avoid these problems by choosing function names carefully:

\begin{itemize}

\item Use long, descriptive names for functions, not single letters like {\tt f}.

\item To be even clearer, use function names that end in {\tt func}.

\item Before you define a function, check whether MATLAB already has a function with the same name.

\end{itemize}




\section{Debugging your head}

When you are working with a new function or a new language feature
for the first time, you should test it in isolation before you
put it into your program.

\index{debugging}

For example, suppose you know that {\tt x} is the sine of some
angle and you want to find the angle.  You find the MATLAB function
{\tt asin}, and you are pretty sure it computes the inverse sine
function.  Pretty sure is not good enough; you want to be very sure.

Since we know $\sin 0 = 0$, we could try

\begin{code}
>> asin(0)
ans = 0
\end{code}

which is correct.  Now, we also know that the sine of 90 degrees is
1, so if we try {\tt asin(1)}, we expect the answer to be 90, right?

\begin{code}
>> asin(1)
ans = 1.5708
\end{code}

Oops.  We forgot that the trig functions in MATLAB work in radians,
not degrees.  So the correct answer is $\pi/2$, which we can
confirm by dividing through by {\tt pi}:

\begin{code}
>> asin(1) / pi
ans = 0.5000
\end{code}

With this kind of testing, you are not really checking for
errors in MATLAB, you are checking your understanding.  If you
make an error because you are confused about how MATLAB works, it
might take a long time to find, because when you look at the code,
it looks right.

\index{debugging!Seventh Theorem}

Which brings us to the Seventh Theorem of Debugging:

\begin{quote}
The worst bugs aren't in your code; they are in your head.
\end{quote}


\section{Glossary}

\begin{description}

\item[analytic solution:] A way of solving an equation by performing
algebraic operations and deriving an explicit way to
compute a value.

\item[numerical solution:] A way of solving an equation by finding
a numerical value that satisfies the equation, often approximately.

\item[numerical method:] A method (or algorithm) for generating
a numerical solution.

\item[zero (of a function):] An argument that makes the result of a function $0$.

\item[function handle:] In MATLAB, a function handle is a way of
referring to a function by name (and passing it as an argument)
without calling it.

\item[shadow:] A kind of name collision in which a new definition
causes an existing definition to become invisible.  In MATLAB,
variable names can shadow built-in functions (with hilarious results).

\end{description}


\section{Exercises}

\begin{ex}

\begin{enumerate}

\index{Chebyshev polynomial}
\index{ezplot@{\tt ezplot}}

\item Write a function called {\tt cheby6} that evaluates the
6th Chebyshev polynomial.  It should take an input variable,
$x$, and return

\begin{equation}
32 x^6 - 48 x^4 + 18 x^2 - 1
\end{equation}

\item Use {\tt ezplot} to display a graph of this function in the
interval from -1 to 1.  Estimate the location of any zeros in this
range.

\item Use {\tt fzero} to find as many different roots as you can.
Does {\tt fzero} always find the root that is closest to the initial
guess?

\end{enumerate}

% cheby6.m
\end{ex}


\begin{ex}
\label{duck}

When a duck is floating on water, how much of its body is submerged?

\index{duck}
\index{sphere}

To estimate a solution to this problem\footnote{This example is adapted from Gerald and Wheatley,
{\em Applied Numerical Analysis}, Fourth Edition, Addison-Wesley,
1989.}, we'll assume that the submerged part of a duck is well approximated by a section of a sphere.
If a sphere with radius $r$ is submerged in water to a depth $d$, the
volume of the sphere below the water line is

\[ V = \frac{\pi}{3} (3r d^2 - d^3) \quad
\mbox{as long as} \quad d < 2 r  \]

We'll also assume that the density of a duck is $\rho$, is $0.3 g / cm^3$ (0.3 times the density of water), and that its mass is $\frac{4}{3} \pi r^3 \rho$.

Finally, according to the law of buoyancy, an object floats at the level where the weight of the displaced water equals the total weight of the object.

\index{density}
\index{buoyancy}

Here are some suggestions for how to proceed:

\begin{itemize}

\item Write an equation relating $\rho$, $d$, and $r$.

\item Rearrange the equation so the right-hand side is zero.
Our goal is to find values of $d$ that are roots of this equation.

\item Write a MATLAB function that evaluates this function.  Test it,
   then make it a quiet function.

\item Make a guess about the value of $d_0$ to use as a starting place.

\item Use {\tt fzero} to find a root near $d_0$.

\item Check to make sure the result makes sense.  In particular,
   check that $d < 2 r$, because otherwise the volume equation
   doesn't work!

\item Try different values of $\rho$ and $r$ and see if you get the
  effect you expect.  What happens as $\rho$ increases?  Goes to
  infinity?  Goes to zero?  What happens as $r$ increases?  Goes to
  infinity?  Goes to zero?

\end{itemize}

% duck.m
\end{ex}

% for another time, figure out how to use fzero to find zeros of...

%\item The Riemann zeta function can be written

%\[ zeta \equiv w \to sum_{k=1}^\infty k^w \]

%where $w$ is a complex number.  If you are not familiar with
%complex numbers, you should skip this problem.



% TODO: Find places in the book where error messages are getting syntax highlighting, and put them in a verbatim environment.

