\chapter{Conditionals}

In this chapter we'll use functions and a new feature, conditional statements, to search for Pythagorean triples.

A Pythagorean triple is a set of integers, like 3, 4, and 5,
that are the lengths of the sides of a right triangle.  Mathematically, it's a set of integers, $a$, $b$, and $c$ such that $a^2 + b^2 = c^2$.

This example will also demonstrate a way to write programs called {\em incremental development}.

\index{Pythagorean triple}

\section{Relational Operators}
\index{operator!relational}

Suppose we have three variables, {\tt a}, {\tt b}, and {\tt c}, and we want to check whether they form a Pythagorean triple.  We can use the equality operator, {\tt ==}, to compare two values:

\begin{code}
>> a = 3;
>> b = 4;
>> c = 5;
>> a^2 + b^2 == c^2

ans = logical 1
\end{code}

The result is a {\em logical} value, which means it's either 1, which means ``true'', or 0, which means ``false''.  Here's an example where the result is false:

\begin{code}
>> c = 6;
>> a^2 + b^2 == c^2
ans = logical 0
\end{code}

It's a common error to use the assignment operator, {\tt =}, instead of the equality operator, {\tt ==}.  If you do, you get an error:

\begin{code}
>> a^2 + b^2 = c^2
 a^2 + b^2 = c^2
           |
Error: Incorrect use of '=' operator. 
To assign a value to a variable, use '='. 
To compare values for equality, use '=='.
\end{code}

The equality operator is one of several {\em relational operators}, so-called because they test relations between values.

For example, {\tt x < 10} is true (1) if the value of {\tt x} is less than 10, or false (0) otherwise.  And {\tt x > 0} is true if {\tt x} is greater than 0.

The other relational operators are {\tt <=} for ``less or equal'', {\tt >=}, for ``greater or equal'', and  \verb+~=+, for ``not equal.''


\section{if Statement}

\index{if statement@{\tt if} statement}
\index{conditional statement}

Now suppose that when we find a Pythagorean triple we want to display a message.
The {\tt if} statement allows you to check for certain conditions
and execute statements if the conditions are met.  For example:

\begin{code}
if a^2 + b^2 == c^2
    disp("Yes, that is a Pythagorean triple.")
end
\end{code}

The syntax is similar to a {\tt for} loop.  The first line
specifies the condition we're interested in.  If the condition is true, 
MATLAB executes the {\em body} of the statement, which is the indented sequence of
statements between the {\tt if} and the {\tt end}.

\index{indentation}

MATLAB doesn't require you to indent the body of an {\tt if}
statement, but it makes your code more readable, so you should do it.

If the condition is not satisfied, the statements in the body are
not executed. 

Sometimes there are alternative statements to
execute when the condition is false.  In that case you can extend
the {\tt if} statement with an {\tt else} clause.

\index{else clause@{\tt else} clause}

The complete version of the previous example might look like this:

\begin{code}
if a^2 + b^2 == c^2
    disp("Yes, that is a Pythagorean triple.")
else
    disp("No, that is not a Pythagorean triple.")
end
\end{code}

Statements like {\tt if} and {\tt for} that contain other statements
are called \emph{compound} statements.  All compound statements end
with... {\tt end}.

\index{compound statement}
\index{statement!compound}



\section{Incremental Development}
\label{increxample}
\index{incremental development}

Now that we have relational operators and {\tt if} statements, let's start writing
the program.

Here are the steps we will follow to develop the program incrementally:

\begin{enumerate}

\item Encapsulate the if statement from the previous section in a function called \verb"is_pythagorean".

\item Write a script named \verb"find_triples" and start with a loop that enumerates values of {\tt a} and displays them.

\item Write a second loop that enumerates values of $b$, and a third loop that enumerates values of $c$.

\item Use \verb"is_pythagorean" to check whether {\tt a}, {\tt b}, and {\tt c} form a Pythagorean triple.

\item Use an {\tt if} to display only values that pass the test.

\item Transform the script into a function and make it take an input variable that specifies the range to search.

\end{enumerate}

Along the way, we'll optimize the program to eliminate unnecessary work.


\section{Logical Functions}

The first step is to create a {\em logical function}, which is a function that returns a logical value.
The following function takes three input variables, {\tt a}, {\tt b}, and {\tt c}, and returns true (1) if they form a Pythagorean triple and false (0) otherwise.

\begin{code}
function res = is_pythagorean(a, b, c)
    if a^2 + b^2 == c^2
        res = 1;
    else
        res = 0;
    end
end
\end{code}

We can use this function like this:

\begin{code}
>> is_pythagorean(3, 4, 5)
ans = 1
\end{code}

But we can write the same function more concisely, like this:

\begin{code}
function res = is_pythagorean(a, b, c)
    res = a^2 + b^2 == c^2;
end
\end{code}

The result of the equality operator is a logical value, which we can assign directly 
to {\tt res}.

I'll put this function is in a file called \verb"is_pythagorean.m", so we can use it as part of our program.


\section{Nested loops}

The next step is to write loops that enumerate different values of {\tt a}, {\tt b}, and 
{\tt c}.  I'll create a new file called \verb"find_triples.m" where I'll develop the rest of the program.

\index{nested loop}
\index{loop!nested}

And I'll start with a loop for {\tt a}:

\begin{code}
for a=1:3
    a
end
\end{code}

It might seem silly to start with such a simple program, but this is an essential element of incremental development: start simple and test as you go.

The output is as expected.

\begin{code}
1
2
3
\end{code}

Now I'll add a second loop for {\tt b}.  It might be tempting to write something like this:

\begin{code}
for a=1:3
    disp(a)
end
for b=1:4
    disp(b)
end
\end{code}

But that loops through the values of {\tt a} and then loops through the values of {\tt b}, and that's not what we want.

Instead, we want to consider every possible pair of values, like this:

\begin{code}
for a=1:3
    for b=1:4
        disp([a,b])
    end
end
\end{code}

Now one loop is inside the other.  The inner loop gets executed 3 times, once for each value of {\tt a}, so here's what the output looks like (I adjusted the spacing to make
the structure clear):

\begin{code}
>> find_triples
     1     1
     1     2
     1     3
     1     4
     2     1
     2     2
     2     3
     2     4
     3     1
     3     2
     3     3
     3     4
\end{code}

The left column shows the values of {\tt a} and the right column shows the values of {\tt b}.

The next step is to search for values of {\tt c} that might make a Pythagorean triple.  The largest possible value for {\tt c} is {\tt a+b}, because otherwise we couldn't form a triangle 
(see \url{https://en.wikipedia.org/wiki/Triangle_inequality}).

\begin{code}
for a=1:3
    for b=1:4
        for c=1:a+b
            disp([a,b,c])
        end
    end
end
\end{code}

After each small change, I run the program again and check the output.

\section{Putting it together}

Now instead of displaying all of the triples, I'll add an {\tt if} statement and display only Pythagorean triples:

\begin{code}
for a=1:3
    for b=1:4
        for c=1:a+b
            if is_pythagorean(a, b, c)
                disp([a,b,c])
            end
        end
    end
end
\end{code}

The result is just one triple:

\begin{code}
>> find_triples
     3     4     5
\end{code}

You might notice that we're wasting some effort here.
After checking $a=1$ and $b=2$, there's no point in checking
$a=2$ and $b=1$.  We can save the extra work by adjusting the
range of {\tt b}.

\begin{code}
for b=a:4
\end{code}

We can save even more work by adjusting the range of {\tt c}.

\begin{code}
for c=b:a+b
\end{code}

Here's the final version:

\begin{code}
for a=1:3
    for b=a:4
        for c=b:a+b
            if is_pythagorean(a, b, c)
                disp([a,b,c])
            end
        end
    end
end
\end{code}

\section{Encapsulation and Generalization}

As a script, this program has the side-effect of assigning values to
{\tt a}, {\tt b}, and {\tt c}, which would be bad if any of those names were in use.  
By wrapping the code in a function, we can avoid name collisions; this process is called {\em encapsulation} because it isolates this program from the workspace.

\index{encapsulation}
\index{generalization}

The first draft of the function takes no input variables:

\begin{code}
function res = find_triples()
    for a=1:3
        for b=a:4
            for c=b:a+b
                if is_pythagorean(a, b, c)
                    disp([a,b,c])
                end
            end
        end
    end
end
\end{code}

The empty parentheses in the signature are not necessary, but
they make it apparent that there are no input variables.  Similarly,
when I call the new function, I like to use parentheses to remind me
that it's a function, not a script:

\begin{code}
>> find_triples()
\end{code}

The output variable isn't necessary, either; it
never gets assigned a value.  But I put it there as a matter of
habit, and also so my function signatures all have the same structure.

\index{output variable}
\index{variable!output}

The next step is to generalize this function by adding input
variables.  The natural generalization is to replace the constant
values 3 and 4 with a variable so we can search an arbitrarily large
range of values.

\begin{code}
function res = find_triples(n)
    for a=1:n
        for b=a:n
            for c=b:a+b
                if is_pythagorean(a, b, c)
                    disp([a,b,c])
                end
            end
        end
    end
end
\end{code}

Here are the results for the range from 1 to 15:

\begin{code}
>> find_triples(15)
     3     4     5
     5    12    13
     6     8    10
     8    15    17
     9    12    15
\end{code}

The triples $5,12,13$ and $8,15,17$ are new, but the others are just multiples of the $3,4,5$ triangle.

\section{Adding a continue Statement}

\index{continue@{\tt continue}}

As a final improvement, let's modify the function so it only
displays the ``lowest'' of each Pythagorean triple, and not the
multiples.

The simplest way to eliminate the multiples is to check whether
$a$ and $b$ share a common factor.  If they do, dividing both
by the common factor yields a smaller, similar triangle that has
already been checked.

\index{gcd function@{\tt gcd} function}
\index{function!{\tt gcd}}

MATLAB provides a {\tt gcd} function that computes the greatest common
divisor of two numbers.  If the result is greater than 1,
$a$ and $b$ share a common factor and we can use the {\tt continue}
statement to skip to the next pair. Listing~\ref{lst:triples_function} contains the final version of this function:

\begin{lstlisting}[caption={Our final Pythagorean triples function}, label={lst:triples_function}] 
function res = find_triples(n)
    for a=1:n
        for b=a:n
            for c=b:a+b
                if gcd(a,b) > 1
(*\codewingding{1}*)	                continue
                end
                if is_pythagorean(a, b, c)
                    disp([a,b,c])
                end
            end
        end
    end
end
\end{lstlisting}

The {\tt continue} statement \wingding{1} causes the program to end the current iteration
immediately, jump to the top of the loop, and ``continue'' with the next iteration.

In this case, since there are three loops, it might not be obvious which loop to jump to, but the rule is to jump to the inner-most loop (which is what we want).

Here are the results with {\tt n=40}:

\begin{code}
>> find_triples(40)
     3     4     5
     5    12    13
     7    24    25
     8    15    17
     9    40    41
    12    35    37
    20    21    29
\end{code}


\section{How Functions Work}
\index{function}

Let's review the sequence of steps that occur when you call a function:

\begin{enumerate}

\item Before the function starts running, MATLAB creates a new
workspace for it.
\index{workspace}

\item MATLAB evaluates each of the arguments and assigns
the resulting values, in order, to the input variables (which
live in the {\em new} workspace).

\item The body of the code executes.  Somewhere in the body
a value gets assigned to the output variable.

\item The function's workspace is destroyed; the only thing
that remains is the value of the output variable and any side
effects the function had (like displaying values).

\item The program resumes from where it left off.  The value
of the function call is the value of the output variable.

\end{enumerate}

When you're reading a program and you come to a function call,
there are two ways to interpret it. You can think about the mechanism I just described,
and follow the execution of the program into the function and back, or you can assume that the function works correctly, and go on to the next statement after the function call.

When you use built-in functions, it's natural to assume that it work, in part because you don't
usually have access to the code in the body of the function.

But when you start writing your own functions, you might
find yourself following the ``flow of execution''.  This can
be useful while you are learning, but as you gain experience, you
should get more comfortable with the idea of writing a function,
testing it to make sure it works, and then forgetting about the
details of how it works.

\index{flow of execution}
\index{abstraction}

Forgetting about details is called {\em abstraction}; in the context
of functions, abstraction means forgetting about {\em how} a function
works, and just assuming (after appropriate testing) that it works.

For many people, it takes some time to get comfortable with functions.  If you are one of them, you might be tempted to avoid functions, and sometimes you can get by without them.

But experienced programmers use functions extensively, for several good reasons. First, each function has its own workspace, so using functions helps
avoid name collisions. 

Functions also lend themselves to incremental development: you can
debug the body of the function first (as a script), then encapsulate
it as a function, and then generalize it by adding input variables.

Also, functions allow you to divide a large problem into small
pieces, work on the pieces one at a time, and then assemble a
complete solution. 

Once you have a function working, you can forget about the
details of how it works and concentrate on what it does.  This
process of abstraction is an important tool for managing the
complexity of large programs.


\section{Summary}

In this chapter, I presented relational operators and {\tt if} statements, and we used them to develop a program that searches for Pythagorean triples.

This example demonstrates a way of developing programs gradually, adding just a few lines of code at a time, and testing as you go.  If you develop programs this way, you will have fewer bugs, and you will find them more quickly.


\section{Glossary}

\begin{description}

\item[side-effect:] An effect, like modifying the workspace, that
is not the primary purpose of a script.

\item[input variable:] A variable in a function that gets its value,
when the function is called, from one of the arguments.

\item[output variable:] A variable in a function that is used to
return a value from the function to the caller.

\item[signature:] The first line of a function definition, which
specifies the names of the function, the input variables and the
output variables.

\item[silent function:] A function that doesn't display anything
or generate a figure, or have any other side-effects.

\item[logical function:] A function that returns a logical value
(1 for ``true'' or 0 for ``false'').

\item[encapsulation:] The process of wrapping part of a program in
a function in order to limit interactions (including name collisions)
between the function and the rest of the program.

\item[generalization:] Making a function more versatile by replacing
specific values with input variables.

\item[abstraction:] The process of ignoring the details of how
a function works in order to focus on a simpler model of what the
function does.

\end{description}


\section{Exercises}

\begin{ex}
\index{Fibonacci number}
\index{Pythagorean triple}

There is an interesting connection between Fibonacci numbers and
Pythagorean triples.  If $F$ is a Fibonacci sequence,

\begin{equation}
(F_i F_{i+3}, 2 F_{i+1} F_{i+2}, F_{i+1}^2 + F_{i+2}^2 )
\end{equation}

is a Pythagorean triple, for all $i \ge 1$.

Write a function named {\tt fib\_triple} that
takes {\tt n} as an input variable, computes 
the first {\tt n} Fibonacci numbers, stores them in a vector,
and checks whether this formula produces Pythagorean triples for numbers in the sequence.

% fib_triple.m
\end{ex}


