

% Leftover material that's been cut.

But sometimes 
if the error messages make
suggestions about what to change,
it can be risky to follow those suggestions.  If you do what you're told, you
might make the error messages go away, but that doesn't mean the
program will do the right thing.  MATLAB doesn't know what the program
is {\em supposed} to do, but you should.

And that brings us to the Eighth Theorem of Debugging:

\begin{quote}
\displaythrm{8}
\end{quote}


\section{Dot and cross products}

Multiplying a vector by a scalar is a straightforward operation;
so is adding two vectors.  But multiplying two vectors is more
subtle.  It turns out that there are two vector operations that
resemble multiplication: the {\bf dot product}
and the {\bf cross product}.

The dot product of vectors $\vec{u}$ and $\vec{v}$ is a scalar:\index{Spatial vectors!dot product}

\begin{equation}
    \vec{u} \cdot \vec{v} = \norm{\vec{u}} \norm{\vec{v}} \cos \theta
\end{equation}

where $\theta$ is the smallest angle between $\vec{u}$ and
$\vec{v}$. We already know how to compute magnitudes, and you could
probably figure out how to compute $\theta$, but you don't have to.
MATLAB provides a function, {\tt dot}, that computes dot products:

\begin{code}
d = dot(U, V)
\end{code}

{\tt dot} works in any number of dimensions, as long as {\tt U}
and {\tt V} have the same number of elements.

If one of the operands is a unit vector, you can use the dot
product to compute the component of a vector $\vec{u}$ that is in
the direction of a unit vector, $\uvec{r}$:

\begin{code}
s = dot(U, Rhat)
\end{code}

In this example, $s$ is the {\bf scalar projection} of $\vec{u}$
onto $\uvec{r}$.  The {\bf vector projection} is the vector
that has magnitude $s$ in the direction of $\uvec{r}$:

\begin{code}
V = dot(U, Rhat) * Rhat
\end{code}

The cross product of vectors $\vec{u}$ and $\vec{v}$ is a vector whose
direction is perpendicular to $\vec{u}$ and $\vec{v}$
(with orientation determined by the right-hand
rule\footnote{\url{https://greenteapress.com/matlab/right}})
and whose magnitude is\index{Spatial vectors!cross product}

\begin{equation}
    \norm{\vec{u} \times \vec{v}} = \norm{\vec{u}} \norm{\vec{v}} \sin \theta
\end{equation}

where (again) $\theta$ is the smallest angle between $\vec{u}$ and
$\vec{v}$. MATLAB provides a function, {\tt cross}, that computes cross
products.

\begin{code}
C = cross(U, V)
\end{code}

{\tt cross} calculates the cross product of corresponding vectors along
the first array dimension whose size equals 3.

A common use of {\tt cross} is to compute torques.  If you represent
a moment arm $\vec{R}$ and a force $\vec{F}$ as vectors with size equal
to 3, then the torque is just

\begin{code}
Tau = cross(R, F)
\end{code}

If the components of {\tt R} are in meters and the components
of {\tt F} are in Newtons, then the torques in {\tt Tau} are
in Newton-meters.


% \section{Scaling}

% \section{Polar coordinates}

\section{What is a model for?}

In Section~\ref{sect:modeling} I defined a ``model'' as a simplified
description of a physical system, and said that a good model
lends itself to analysis and simulation, and makes predictions
that are good enough for the intended purpose.

Since then, we have seen a number of examples; now we can
say more about what models are for.  The goals of a model tend
to fall into three categories.

\begin{description}

\item[prediction:] Some models make predictions about physical
systems.  As a simple example, the duck model in
Exercise~\ref{ex:duck} predicts the level a duck floats at.  At the other
end of the spectrum, global climate models try to predict the weather
tens or hundreds of years in the future.

\item[design:] Models are useful for engineering design, especially
for testing the feasibility of a design and for optimization.  For
example, in Exercise~\ref{ex:golf} you were asked to design the golf
swing with the perfect combination of launch angle, velocity and spin.

\item[explanation:] Models can answer scientific questions.  For
example, the Lotka-Volterra model in Section~\ref{sect:lotka} offers a
possible explanation of the dynamics of animal populations systems in
terms of interactions between predator and prey species.

\end{description}

The exercises at the end of this chapter include one model of
each type.


\section{Glossary}

\begin{description}

\item[dot product:] A scalar product of two vectors, proportional to the
norms of the vectors and the cosine of the smallest angle between them.

\item[cross product:] A vector product of two vectors with norm
proportional to the norms of the vectors and the sine of the angle
between them, and direction perpendicular to both.

\item[projection:] The component of one vector that is in the
direction of the other (might be used to mean ``scalar projection'' or
``vector projection'').

\end{description}


\chapter{Nested Functions}
\label{chpt:nestedfuns}
\index{Functions!nested}

This appendix provides an alternative way to structure code that
involves helper functions, which were discussed in Section~\ref{sect:funfiles};
there, we saw an example of an M-file with more than one function:

\begin{code}
function res = duck()
    error = error_func(10)
end

function res = error_func(h)
    rho = 0.3;      % density in g / cm^3
    r = 10;         % radius in cm
    res = ...
end
\end{code}

Because the first function ends before the second begins, these
functions are at the same level of indentation.  Functions like
these are parallel, as opposed to nested.  A nested function is
defined inside another, like this:

\begin{code}
function res = duck()
    error = error_func(10)

    function res = error_func(h)
        rho = 0.3;      % density in g / cm^3
        r = 10;         % radius in cm
        res = ...
    end
end
\end{code}

The function {\tt duck} is the enclosing function and {\tt
error\_func} is the nested function.

Nesting functions is useful because the variables of the outer
function can be accessed from the inner function.  This is not
possible with parallel functions.

In this example, using a nested function makes it possible to
move the parameters {\tt rho} and {\tt r} out of {\tt error\_func}.

\begin{code}
function res = duck(rho)
    r = 10;
    error = error_func(10)

    function res = error_func(h)
        res = ...
    end
end
\end{code}

Both {\tt rho} and {\tt r} can be accessed from {\tt error\_func}.
By making {\tt rho} an input argument, we made it easier to test
{\tt duck} with different parameter values.

Nested functions can also help de-clutter code (but not always). For
more information on nested functions, please consult the official documentation
at \url{https://greenteapress.com/matlab/nested}.
