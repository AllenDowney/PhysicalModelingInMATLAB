\documentclass[main.tex]{subfiles}

\begin{document}

\chapter{Vectors}
\label{chpt:vectors}

This chapter presents conditional statements and vectors, and three patterns for working with data: reduce, apply, and search.


\section{Checking preconditions}

\index{precondition}
\index{postcondition}

Some of the loops in the previous chapter don't work
if the value of {\tt n} isn't set correctly before the loop runs.
For example, this loop computes the sum of the first {\tt n} elements
of a geometric sequence:

\begin{code}
A1 = 1;
total = 0;
for i=1:n
    a = A1 * 0.5^(i-1);
    total = total + a;
end
ans = total
\end{code}

It works for any positive value of {\tt n}, but what if {\tt n}
is negative?  In that case, you get:

\begin{code}
total = 0
\end{code}

Why?  Because the expression {\tt 1:-1} means ``all the numbers
from 1 to -1, counting up by 1.''  
It's not immediately obvious
what that should mean, but MATLAB's interpretation is that there
aren't any numbers that fit that description, so the result is

\index{colon operator}
\index{operator!colon}

\begin{code}
>> 1:-1
ans = 1x0 empty double row vector
\end{code}

This indicates that the result is an empty row vector of 
double-precision numbers.  An empty row vector has one row and zero columns.  We will learn more about MATLAB vectors later in this
chapter.

\index{vector}
\index{range}

In any case, if you loop over an empty range,
the loop never runs at all, which is why in this example the
value of {\tt total} is zero for any negative value of {\tt n}.

If you are sure that you will never make a mistake, and that the
preconditions of your functions will always be satisfied, then you
don't have to check.  But for the rest of us, it is dangerous to write
a script, like this one, that quietly produces the wrong answer (or
at least a meaningless answer) if the input value is negative.
A better alternative is to use an {\tt if} statement.


\section{if statements}

\index{if statement@{\tt if} statement}
\index{conditional statement}

The {\tt if} statement allows you to check for certain conditions
and execute statements if the conditions are met.  In the previous
example, we could write:

\begin{code}
if n<0
    ans = NaN
end
\end{code}

The syntax is similar to a {\tt for} loop.  The first line
specifies the condition we are interested in; in this case we
are asking if {\tt n} is negative.  If it is, MATLAB executes
the body of the statement, which is the indented sequence of
statements between the {\tt if} and the {\tt end}.

\index{indentation}

MATLAB doesn't require you to indent the body of an {\tt if}
statement, but it makes your code more readable, so you should do it.
Don't make me tell you again.

In this example, the ``right'' thing to do if {\tt n} is negative
is to set {\tt ans = NaN}, which is a standard way to indicate that
the result is undefined (not a number).

\index{NaN}
\index{not a number}

If the condition is not satisfied, the statements in the body are
not executed.  Sometimes there are alternative statements to
execute when the condition is false.  In that case you can extend
the {\tt if} statement with an {\tt else} clause.

\index{else clause@{\tt else} clause}

The complete version of the previous example might look like this:

\begin{code}
if n<0
    ans = NaN
else
    A1 = 1;
    total = 0;
    for i=1:n
        a = A1 * 0.5^(i-1);
        total = total + a;
    end
    ans = total
end
\end{code}

Statements like {\tt if} and {\tt for} that contain other statements
are called {\bf compound} statements.  All compound statements end
with... {\tt end}.

\index{compound statement}
\index{statement!compound}

In this example, one of the statements in the {\tt else} clause is a
{\tt for} loop.  Putting one compound statement inside another is
legal and common, and sometimes called {\bf nesting}.

\index{nesting}


\section{Relational operators}
\index{Operations!relational}

The operators that compare values, like {\tt <} and {\tt >} are
called {\bf relational operators} because they test the relationship
between two values.  The result of a relational operator is one
of the {\bf logical values}:
either 1, which represents ``true,''  or 0, which represents ``false.''

\index{relational operator}
\index{operator!relational}
\index{logical value}
\index{value!logical}

Relational operators often appear in {\tt if} statements, but you can also evaluate them at the prompt:

\begin{code}
>> x = 5;
>> x < 10
ans = 1
\end{code}

You can assign a logical value to a variable:

\begin{code}
>> flag = x > 10
flag = 0
\end{code}

A variable that contains a logical value is often called a {\bf flag}
because it flags the status of some condition.

\index{flag}
\index{variable!flag}

The other relational operators are {\tt <=} and {\tt >=}, which are
self-explanatory, {\tt ==}, for ``equal,'' and
\verb+~=+, for ``not equal.''

Don't forget that {\tt ==} is the operator that tests equality,
and {\tt =} is the assignment operator.  If you try to use {\tt =} in
an {\tt if} statement, you get the following error:

\index{equality}
\index{assignment}

\begin{code}
>> if x=5
 if x=5
     |
Error: Incorrect use of '=' operator. 
To assign a value to a variable, use '='.
To compare values for equality, use '=='.
 
Did you mean:
>> x = 5
\end{code}

In this case, the error message is pretty helpful.


\section{Logical operators}
\label{sect:logop}

\index{logical operator}
\index{operator!logical}
\index{interval}

To test if a number falls in an interval, you might be
tempted to write something like {\tt 0 < x < 10}, but that
would be wrong, so very wrong.  Unfortunately, in many cases,
you will get the right answer for the wrong reason.  For
example:

\begin{code}
>> x = 5;
>> 0 < x < 10            % right for the wrong reason
ans = 1
\end{code}

But don't be fooled!

\begin{code}
>> x = 17
>> 0 < x < 10            % just plain wrong
ans = 1
\end{code}

The problem is that MATLAB is evaluating the operators from left
to right, so first it checks if {\tt 0<x}.  It is, so the result
is 1.  Then it compares the logical value 1 (not the value of
{\tt x}) to 10.  Since {\tt 1<10}, the result is true, even though
{\tt x} is not in the interval.

For beginning programmers, this is an evil, evil bug!

\index{nesting}

One way around this problem is to use a nested {\tt if} statement to
check the two conditions separately:

\begin{code}
ans = 0
if 0<x
    if x<10
        ans = 1
    end
end
\end{code}

But it is more concise to use the AND operator, {\tt \&\&}, to
combine the conditions.

\begin{code}
>> x = 5;
>> 0<x && x<10
ans = 1

>> x = 17;
>> 0<x && x<10
ans = 0
\end{code}

The result of AND is true if {\em both} of the operands are
true.  The OR operator, {\tt ||}, is true if {\em either or both}
of the operands are true.

\index{AND operator}
\index{operator!AND}

\index{OR operator}
\index{operator!OR}


\section{Vectors}

The values we have seen so far have mostly been single numbers,
which are called {\bf scalars} to contrast them with {\bf vectors}
and {\bf matrices}, which are collections of numbers.

\index{scalar}
\index{vector}
\index{matrix}

A vector in MATLAB is similar to a sequence in mathematics;
it is an ordered set of numbers that is indexed by positive integers.
There are several ways to create vectors; one of the most common is
to put a sequence of numbers in square brackets:

\begin{code}
>> [1 2 3]
ans = 1     2     3
\end{code}

In general, anything you can do with a scalar, you can also do with
a vector.  You can assign a vector to a variable:

\begin{code}
>> X = [1 2 3]
X = 1     2     3
\end{code}

In this book, we often capitalize single-letter variables that hold vectors 
(and matrices).
That's just a convention; MATLAB doesn't require it, but especially for beginning programmers, it is a useful way to distinguish vectors (and matrices) from scalars.

\index{variable name}


\section{Vector arithmetic}

\index{vector arithmetic}
\index{arithmetic!vector}

You can perform arithmetic with vectors, too.  If you add a scalar
to a vector, MATLAB increments each element of the vector:

\begin{code}
>> Y = X + 5
Y = 6     7     8
\end{code}

The result is a new vector; the original value of {\tt X} is not
changed.

If you add two vectors, MATLAB adds the corresponding elements of each
vector and creates a new vector that contains the sums:

\begin{code}
>> Z = X+Y
Z = 7     9    11
\end{code}

But adding vectors only works if the operands are the same size.
Otherwise you get an error:

\begin{code}
>> W = [1 2]
W = 1     2     3

>> X + W
Matrix dimensions must agree.
\end{code}

The error message in this case is confusing, because we are thinking
of these values as vectors, not matrices.  The problem is a slight
mismatch between math vocabulary and MATLAB vocabulary.


\section{Everything is a matrix}

In math (specifically in linear algebra) a vector is a one-dimensional
sequence of values and a matrix is two-dimensional. And, if you want
to think of it that way, a scalar is zero-dimensional.

\index{matrix}
\index{linear algebra}
\index{whos command@{\tt whos} command}
\index{command!{\tt whos}}

In MATLAB, the values we have seen so far are all two-dimensional matrices (except strings).
You can see this if you use the {\tt whos} command to display the
values in the workspace.  {\tt whos} is similar to {\tt who}, but it also displays the size of each value and other information.

\index{variable}

To demonstrate, I'll remove any existing variables and then 
make a scalar, a vector, and a matrix:

\begin{code}
>> clear
>> scalar = 5
scalar = 5

>> vector = [1 2 3 4 5]
vector = 1     2     3     4     5

>> matrix = ones(2,3)
matrix =
     1     1     1
     1     1     1
\end{code}

The built-in function {\tt ones} builds a new matrix with the given
number of rows and columns, and sets all the elements to 1.
Now let's see what we've got.

\index{ones@{\tt ones}}

\begin{code}
>> whos
  Name        Size            Bytes  Class     Attributes
              
  scalar      1x1                 8  double              
  vector      1x5                40  double              
  matrix      2x3                48  double              
\end{code}

According to MATLAB, all these variables are of class {\tt double}, which
is another name for a double-precision floating-point number.

\index{floating-point number}
\index{number!floating-point}
\index{double precision}

But they have difference sizes: 

\begin{itemize}

\item The size of {\tt scalar} is {\tt 1x1}, which means it has 1 row and 1 column.  

\item {\tt vector} has 1 row and 5 columns.  

\item And {\tt matrix} has 2 rows and 3 columns.

\end{itemize}

The point of all this is that you can think of your variables as
scalars, vectors, and matrices, and I think you should.  But in MATLAB they are all matrices.


\section{Elementwise operators}
\label{elementwise}

\index{elementwise operator}
\index{operator!elementwise}

If you have two vectors with the same length, you can add and subtract them:

\begin{code}
>> X = [1 2 3]
X = 1     2     3

>> Y = [4 5 6]
Y = 1     4     6

>> X + Y
ans = 5     7     9

>> X - Y
ans = -3    -3    -3
\end{code}

These operations are performed {\bf elementwise}; that is, MATLAB adds or subtracts corresponding elements of the two vectors, and the result is a vector with the same size.

But if you divide two vectors, you might be surprised by the result:

\begin{code}
>> X / Y
ans = 0.4156
\end{code}

MATLAB is performing a matrix operation called right division, which I will not try to explain.  If you want to divide the elements of {\tt X} by the elements of {\tt Y}, you have to use {\tt ./}, which is elementwise division:

\begin{code}
>> X ./ Y
ans = 0.2500    0.4000    0.5000
\end{code}

Multiplication has the same problem.  If you use {\tt *}, MATLAB does matrix multiplication.  With these two vectors, matrix multiplication is not defined, and you get an error:

\begin{code}
>> X * Y
Error using  * 
Incorrect dimensions for matrix multiplication. 
Check that the number of columns in the first matrix 
matches the number of rows in the second matrix
To perform elementwise multiplication, use '.*'.
\end{code}

In this case, the error message is pretty helpful.  As it suggests, you can use {\tt .*} to perform elementwise multiplication:

\index{matrix multiplication}
\index{multiplication!matrix}

\begin{code}
>> X .* Y
ans = 4    10    18
\end{code}

As an exercise, see what happens if you use the exponentiation operator,
\verb"^", with a vector.


\section{Indices}

\index{element}
\index{parentheses}

You can select an element from a vector with parentheses:

\begin{code}
>> Y = [6 7 8 9]
Y = 6    7     8     9

>> Y(1)
ans = 6

>> Y(4)
ans = 9
\end{code}

This means that the first element of {\tt Y} is 6 and the
fourth element is 9.
The number in parentheses is called the {\bf index} because it indicates which element of the vector you want.

\index{index}

The index can be any kind of expression.

\begin{code}
>> i = 1;

>> Y(i+1)
ans = 7
\end{code}

We can use a loop to display the elements of {\tt Y}:

\index{loop}

\begin{code}
for i=1:4
     Y(i)
end
\end{code}

Each time through the loop we use a different value of {\tt i}
as an index into {\tt Y}.

\index{length function@{\tt length} function}
\index{function!{\tt length}}

A limitation of this example is that we had to know the number
of elements in {\tt Y}.  We can make it more general by using
the {\tt length} function, which returns the number of elements
in a vector:

\begin{code}
for i=1:length(Y)
     Y(i)
end
\end{code}

Now that works for a vector of any length.


\section{Indexing errors}

\index{index}
\index{expression}

An index can be any kind of expression, but the value of the
expression has to be a positive integer, and it has to be
less than or equal to the length of the vector.  If it's
not a positive integer, you get an error:

\begin{code}
>> Y(0)
Array indices must be positive integers or logical values.
\end{code}

If it's not an integer, you get an error:

\begin{code}
>> Y(1.5)
Array indices must be positive integers or logical values.
\end{code}

If the index is too big, you also get an error:

\begin{code}
>> Y(5)
Index exceeds the number of array elements (4).
\end{code}

The error messages use the word ``array'' rather than ``matrix'', but they mean the same thing, at least for now.

\index{array}


\section{Vectors and sequences}

\index{vector}
\index{sequence}
\index{Fibonacci}

Vectors and sequences go together nicely.
For example, another way to evaluate the Fibonacci sequence is by
storing successive values in a vector.  Again, the definition of the
Fibonacci sequence is $F_1 = 1$, $F_2 = 1$, and $F_{i} = F_{i-1} +
F_{i-2}$ for $i > 2$.  In MATLAB, that looks like

\begin{code}
F(1) = 1
F(2) = 1
for i=3:n
    F(i) = F(i-1) + F(i-2)
end
\end{code}

I use a capital letter for the vector {\tt F}
and lower-case letters for the scalars {\tt i} and {\tt n}.

If you had any trouble with Exercise~\ref{ex:fib2}, you have to
appreciate the simplicity of this version.  The MATLAB syntax is
similar to the math notation, which makes it easier to check
correctness.  

\index{range}
\index{loop}

However, you have to be careful with the range of the loop.
In the previous version, the loop runs from {\tt 3} to {\tt n},
and each time we assign a value to the {\tt i}th element.  

It would also work to ``shift'' the index over by two,
running the loop from 1 to {\tt n-2}:

\begin{code}
F(1) = 1
F(2) = 1
for i=1:n-2
    F(i+2) = F(i+1) + F(i)
end
\end{code}

Either version is fine, but you have to choose one approach
and be consistent.  If you combine elements of both, you will
get confused.  I prefer the version that has {\tt F(i)} on the
left side of the assignment, so that each time through the loop
it assigns the {\tt i}th element.

If you only want the $n$th Fibonacci number, storing
the whole sequence wastes some space.  But if wasting space
makes your code easier to write and debug, that's probably ok.

\begin{ex}
Write a loop that computes the first {\tt n} elements
of the geometric sequence $A_{i+1} = A_i/2$ with $A_1 = 1$.  Notice that
math notation puts $A_{i+1}$ on the left side of the equality.
When you translate to MATLAB, you may want to shift the index.
\end{ex}


\section{Plotting vectors}

\index{plotting vector}
\index{vector!plotting}
\index{plot@{\tt plot}}

If you call {\tt plot} with a vector as an argument,
MATLAB plots the indices on the $x$-axis and the elements on the
$y$-axis.
To plot the Fibonacci numbers we computed in the previous section:

\begin{code}
plot(F)
\end{code}

This display is often useful for debugging, especially
if your vectors are big enough that displaying the elements on
the screen is unwieldy.

\index{style string}
\index{string!style}

By default, MATLAB draws a blue line, but you can override that
setting with a style string, as we saw in Section~\ref{sect:plotting}.
For example, the string {\tt 'ro-'} tells MATLAB to plot a red circle
at each data point; the hyphen means the points should be connected
with a line.



\section{Reduce}
\label{sect:reduce}

A frequent use of loops is to run through the elements of an array
and add them up, or multiply them together, or compute the sum
of their squares, etc.  This kind of operation is called {\bf reduce},
because it reduces a vector with multiple elements down to a single
scalar.

\index{reduce}
\index{vector}
\index{scalar}

For example, this loop adds up the elements of a vector named {\tt X}
(which we assume has been defined).

\begin{code}
total = 0
for i=1:length(X)
    total = total + X(i)
end
ans = total
\end{code}

The use of {\tt total} as an accumulator is similar to what we
saw in Section~\ref{sect:series}.  Again, we use the {\tt length} function
to find the upper bound of the range, so this loop will work
regardless of the length of {\tt X}.
Each time through the loop, we add
in the {\tt i}th element of {\tt X}, so at the end of the loop
{\tt total} contains the sum of the elements.

\index{accumulator}
\index{length function@{\tt length} function}

\section{Apply}
\label{sect:apply}

Another common use of a loop is to run through the elements of
a vector, perform some operation on the elements, and create
a new vector with the results.  This kind of operation is called
{\bf apply}, because you apply the operation to each element in
the vector.

\index{apply}

For example, the following loop computes a vector {\tt Y} that
contains the squares of the elements of {\tt X} (assuming, again,
that {\tt X} is already defined).

\begin{code}
for i=1:length(X)
    Y(i) = X(i)^2
end
\end{code}


\section{Search}
\label{sect:search}

Yet another use of loops is to search the elements of a vector
and return the index of the value you are looking for (or the
first value that has a particular property).  

\index{search}

For example, the following loop finds the index of the element 0 in 
{\tt X}:

\begin{code}
for i=1:length(X)
    if X(i) == 0
        ans = i
    end
end
\end{code}

A funny thing about this loop is that it keeps going after it
finds what it is looking for.  That might be what you want; if the
target value appears more than one, this loop provides the index
of the {\em last} one.

\index{break statement@{\tt break} statement}
\index{statement!{\tt break}}

But if you want the index of the first one (or you know that there
is only one), you can save some unnecessary looping by using the
{\tt break} statement.

\begin{code}
for i=1:length(X)
    if X(i) == 0
        ans = i
        break
    end
end
\end{code}

{\tt break} does pretty much what it sounds like.  It ends the
loop and proceeds immediately to the next statement after the
loop (in this case, there isn't one, so the code ends).


\section{Spoiling the fun}

Experienced MATLAB programmers would never write the kind of loops
in this chapter, because MATLAB provides simpler and faster ways to
perform many reduce, filter and search operations.

\index{sum function@{\tt sum} function}
\index{function!{\tt sum}}
\index{prod function@{\tt prod} function}
\index{function!{\tt prod}}

For example, the {\tt sum} function computes the sum of the elements
in a vector and {\tt prod} computes the product.

Many apply operations can be done with elementwise operators.
The following statement is more concise than the loop in
Section~\ref{sect:apply}

\begin{code}
Y = X .^ 2
\end{code}

And {\tt find} can perform search operations:

\begin{code}
>> X = [3 2 1 0]
X = 3     2     1     0

>> find(X==0)
ans = 4
\end{code}

If you understand loops and you are are comfortable with the
shortcuts, feel free to use them!  Otherwise, you can always write
out the loop.


\section{Name Collisions}
\label{sect:collision}

\index{name collision}
\index{collision!name}
\index{workspace}

All scripts run in the same workspace, so if one script changes the value of a variable, all other scripts see the change.  With a small number of simple scripts, that's not a problem, but eventually the interactions between scripts become unmanageable.

For example, the following (increasingly familiar) script computes the
sum of the first {\tt n} terms in a geometric sequence, but it also
has the {\bf side-effect} of assigning values to {\tt A1}, {\tt total},
{\tt i}, and {\tt a}.

\begin{code}
A1 = 1;
total = 0;
for i=1:10
    a = A1 * 0.5^(i-1);
    total = total + a;
end
ans = total
\end{code}

If you were using any of those variable names before calling this
script, you might be surprised to find, after running the script,
that their values had changed.  If you have two scripts that use
the same variable names, you might find that they work separately
and then break when you try to combine them.  This kind of
interaction is called a {\bf name collision}.

\index{script}

As the number of scripts you write increases, and they get longer
and more complex, name collisions become more of a problem.  Avoiding
this problem is one of the motivations for functions.


\section{Glossary}

\begin{description}

\item[compound statement:] A statement, like {\tt if} and {\tt for}, that
contains other statements in an indented body.

\item[nesting:] Putting one compound statement in the body of another.

\item[relational operator:] An operator that compares two values and
generates a logical value as a result.

\item[logical value:] A value that represents either ``true'' or
``false''.  MATLAB uses the values 1 and 0, respectively.

\item[flag:] A variable that contains a logical value, often used
to store the status of some condition.

\item[scalar:] A single value.

\item[vector:] A sequence of values.

\item[matrix:] A two-dimensional collection of values (also called
``array'' in some MATLAB documentation).

\item[index:] An integer value used to indicate one of the values
in a vector or matrix (also called subscript in some MATLAB documentation).

\item[element:] One of the values in a vector or matrix.

\item[elementwise:] An operation that acts on the individual elements
of a vector or matrix (unlike some linear algebra operations).

\item[reduce:] A way of processing the elements of a vector and
generating a single value; for example, the sum of the elements.

\item[apply:] A way of processing a vector by performing some operation
on each of the elements, producing a vector that contains the
results.

\item[search:] A way of processing a vector by examining the
elements in order until one is found that has the desired property.

\item[name collision:] The scenario where two scripts that use the
same variable name interfere with each other.

\end{description}

\section{Exercises}

\begin{ex}
Write an expression that computes the square root of the sum of the squares of the elements of a vector, without using a loop.

\index{square root}
\end{ex}

\begin{ex}
\label{ex:fibratio}

The ratio of consecutive Fibonacci numbers, $F_{n+1}/F_{n}$, converges
to a constant value as $n$ increases.  Write a script that computes
a vector with the first $n$ elements of a Fibonacci sequence (assuming
that the variable {\tt n} is defined), and then computes a new
vector that contains the ratios of consecutive Fibonacci numbers.
Plot this vector to see if it seems to converge.  What value does
it converge on?

\index{Fibonacci}

% fibonacci4.m
\end{ex}

\begin{ex}
The following set of equations is based on a famous example of a chaotic system, the Lorenz attractor\footnote{See \url{https://en.wikipedia.org/wiki/Lorenz_system}.}:
%
\begin{eqnarray}
x_{i+1} &=& x_i + \sigma \left( y_i - x_i \right) dt  \\
y_{i+1} &=& y_i + \left[ x_i (r - z_i) - y_i \right] dt   \\
z_{i+1} &=& z_i + \left( x_i y_i - b z_i \right) dt
\end{eqnarray}
%
\begin{itemize}

\item Write a script that computes the first 10 elements of the sequences
$X$, $Y$, and $Z$ and stores them in vectors named {\tt X}, {\tt Y},
and {\tt Z}.

Use the initial values $X_1 = 1$, $Y_1 = 2$, and $Z_1 = 3$, with values
$\sigma = 10$, $b = 8/3$, and $r = 28$, and with $dt = 0.01$.

\item Read the documentation for {\tt plot3} and {\tt comet3} and
plot the results in 3 dimensions.

\item Once the code is working, use semi-colons to suppress the output
and then run the program with sequence length 100, 1000, and 10000.

\item Run the program again with different starting conditions.
What effect does it have on the result?

\item Run the program with different values for $\sigma$, $b$, and $r$
and see if you can get a sense of how each variable affects the
system.

\end{itemize}

\index{Lorenz attractor}
% lorenz.m
\end{ex}


\begin{ex}
The logistic map\footnote{See \url{https://en.wikipedia.org/wiki/Logistic_map}} is described by the following equation:

\index{logistic map}

\begin{equation}
X_{i+1} = r X_i (1-X_i)
\end{equation}

where $X_i$ is a number between zero and one and $r$ is a positive number that represents.

\begin{itemize}

\item Write a script named {\tt logmap} that computes the first 50
elements of $X$ with {\tt r=3.9} and {\tt X1=0.5}, where
{\tt r} is the parameter of the logistic map and {\tt X1} is the
initial value.

\item Plot the results for a range of values of $r$ from 2.4 to 4.0.
How does the behavior of the system change as you vary $r$?

\end{itemize}

% logmap.m
\end{ex}


% chap05


\end{document}
