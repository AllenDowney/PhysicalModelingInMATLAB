\section{Working with Conditions}

\index{precondition}
\index{postcondition}

Some of the loops in the previous chapter don't work
if the value of {\tt n} isn't set correctly before the loop runs.
For example, this loop computes the sum of the first {\tt n} elements
of a geometric sequence:

\begin{code}
A1 = 1;
total = 0;
for i=1:n
    a = A1 * 0.5^(i-1);
    total = total + a;
end
ans = total
\end{code}

It works for any positive value of {\tt n}, but what if {\tt n}
is negative?  In that case, you get:

\begin{code}
total = 0
\end{code}

Why?  Because the expression {\tt 1:-1} means ``all the numbers
from 1 to -1, counting up by 1.''  
It's not immediately obvious
what that should mean, but MATLAB's interpretation is that there
aren't any numbers that fit that description, so the result is

\index{colon operator}
\index{operator!colon}
\index{vector}

\begin{code}
>> 1:-1
ans = 1x0 empty double row vector
\end{code}

The result is an empty vector with one row and 0 columns.
If you loop over an empty vector,
the loop never runs at all, which is why in this example the
value of {\tt total} is zero for any negative value of {\tt n}.

If you're sure that you'll never make a mistake, and that the
preconditions of your functions will always be satisfied, then you
don't have to check.  But for the rest of us, it's dangerous to write
a script, like this one, that quietly produces the wrong answer (or
at least a meaningless answer) if the input value is negative.
A better alternative is to use an {\tt if} statement.

\subsection{if Statements}

\index{if statement@{\tt if} statement}
\index{conditional statement}

The {\tt if} statement allows you to check for certain conditions
and execute statements if the conditions are met.  In the previous
example, we could write:

\begin{code}
if n<0
    ans = NaN
end
\end{code}

The syntax is similar to a {\tt for} loop.  The first line
specifies the condition we're interested in; in this case we're asking if {\tt n} is negative.  If it is, MATLAB executes
the body of the statement, which is the indented sequence of
statements between the {\tt if} and the {\tt end}.

\index{indentation}

MATLAB doesn't require you to indent the body of an {\tt if}
statement, but it makes your code more readable, so you should do it.
Don't make me tell you again.

In this example, the ``right'' thing to do if {\tt n} is negative
is to set {\tt ans = NaN}, which is a standard way to indicate that
the result is undefined (not a number).

\index{NaN}
\index{not a number}

If the condition is not satisfied, the statements in the body are
not executed.  Sometimes there are alternative statements to
execute when the condition is false.  In that case you can extend
the {\tt if} statement with an {\tt else} clause.

\index{else clause@{\tt else} clause}

The complete version of the previous example might look like this:

\begin{code}
if n<0
    ans = NaN
else
    A1 = 1;
    total = 0;
    for i=1:n
        a = A1 * 0.5^(i-1);
        total = total + a;
    end
    ans = total
end
\end{code}

Statements like {\tt if} and {\tt for} that contain other statements
are called \emph{compound} statements.  All compound statements end
with... {\tt end}.

\index{compound statement}
\index{statement!compound}

In this example, one of the statements in the {\tt else} clause is a
{\tt for} loop.  Putting one compound statement inside another is
legal and common, and sometimes called \emph{nesting}.

\index{nesting}


\subsection{Relational Operators}
\index{Operations!relational}

The operators that compare values, like {\tt <} and {\tt >} are
called \emph{relational operators} because they test the relationship
between two values.  The result of a relational operator is one
of the \emph{logical values}:
either 1, which represents ``true,''  or 0, which represents ``false.''

\index{relational operator}
\index{operator!relational}
\index{logical value}
\index{value!logical}

Relational operators often appear in {\tt if} statements, but you can also evaluate them at the prompt:

\begin{code}
>> x = 5;
>> x < 10
ans = 1
\end{code}

You can assign a logical value to a variable:

\begin{code}
>> flag = x > 10
flag = 0
\end{code}

A variable that contains a logical value is often called a \emph{flag}
because it flags the status of some condition.

\index{flag}
\index{variable!flag}

The other relational operators are {\tt <=} and {\tt >=}, which are
self-explanatory, {\tt ==}, for ``equal,'' and
\verb+~=+, for ``not equal.''

Don't forget that {\tt ==} is the operator that tests equality,
and {\tt =} is the assignment operator.  If you try to use {\tt =} in
an {\tt if} statement, you get an error:

\index{equality}
\index{assignment}

\begin{code}
>> if x=5
 if x=5
     |
Error: Incorrect use of '=' operator. 
To assign a value to a variable, use '='.
To compare values for equality, use '=='.
 
Did you mean:
>> x = 5
\end{code}

In this case, the error message is pretty helpful.

\subsection{Logical operators}
\label{logop}

\index{logical operator}
\index{operator!logical}
\index{interval}

To test if a number falls in an interval, you might be
tempted to write something like {\tt 0 < x < 10}, but that
would be wrong, so very wrong.  Unfortunately, in many cases,
you'll get the right answer for the wrong reason.  For
example:

\begin{code}
>> x = 5;
>> 0 < x < 10            % right for the wrong reason
ans = 1
\end{code}

But don't be fooled!

\begin{code}
>> x = 17
>> 0 < x < 10            % just plain wrong
ans = 1
\end{code}

The problem is that MATLAB is evaluating the operators from left
to right, so first it checks if {\tt 0<x}.  It is, so the result
is 1.  Then it compares the logical value 1 (not the value of
{\tt x}) to 10.  Since {\tt 1<10}, the result is true, even though
{\tt x} is not in the interval.

For beginning programmers, this is an evil, evil bug!

\index{nesting}

One way around this problem is to use a nested {\tt if} statement to
check the two conditions separately:

\begin{code}
ans = 0
if x>0
    if x<10
        ans = 1
    end
end
\end{code}

But it's more concise to use the AND operator, {\tt \&\&}, to combine the conditions.

\begin{code}
>> x = 5;
>> x>0 && x<10
ans = 1

>> x = 17;
>> x>0 && x<10
ans = 0
\end{code}

The result of AND is true if {\em both} of the operands are
true.  The OR operator, {\tt ||}, is true if {\em either or both}
of the operands are true.



...



\index{range}
\index{loop}

However, you have to be careful with the range of the loop.
In the previous version, the loop runs from {\tt 3} to {\tt n},
and each time we assign a value to the {\tt i}th element.  

It would also work to ``shift'' the index over by two, running the loop from 1 to {\tt n-2}:

\begin{code}
F(1) = 1
F(2) = 1
for i=1:n-2
    F(i+2) = F(i+1) + F(i)
end
\end{code}

Either version is fine, but you have to choose one approach
and be consistent.  If you combine elements of both, you'll
get confused.  I prefer the version in Listing~\ref{lst:fib_vec} that has {\tt F(i)} on the
left side of the assignment, so that each time through the loop
it assigns the {\tt i}th element.


\subsection{Everything is a Matrix}

In math (specifically in linear algebra) a vector is one-dimensional
and a \emph{matrix} is two-dimensional.

\index{matrix}
\index{linear algebra}
\index{whos command@{\tt whos} command}
\index{command!{\tt whos}}

But in MATLAB, everything is a matrix (except strings).
You can see this if you use the {\tt whos} command to display the
variables in the workspace.  {\tt whos} is similar to {\tt who}, but it also displays the size of each value and other information.

\index{variable}

To demonstrate, I'll make one of each kind of value:

\begin{code}
>> number = 5
number = 5

>> vector = [1 2 3 4 5]
vector = 1     2     3     4     5

>> matrix = ones(2,3)
matrix =
     1     1     1
     1     1     1
\end{code}

The built-in function {\tt ones} builds a new matrix with the given
number of rows and columns, and sets all the elements to 1.
Now let's see what we've got.

\index{ones@{\tt ones}}

\begin{code}
>> whos
  Name        Size            Bytes  Class     Attributes

  number      1x1                 8  double              
  vector      1x5                40  double               
  matrix      2x3                48  double              
\end{code}

According to MATLAB, all three are matrices.  They have the same class, {\tt double}, which
means they contain double-precision floating-point numbers.

\index{floating-point number}
\index{number!floating-point}
\index{double precision}

But they have difference sizes: the size of {\tt number} is {\tt 1x1}, which means it has 1 row and 1 column; {\tt vector} has 1 row and 5 columns; and {\tt matrix} has 2 rows and 3 columns.

So, you can think of your values as
numbers, vectors, and matrices, but in MATLAB they are all matrices.



\subsection{Search}
\label{search}

Yet another use of loops is to search the elements of a vector
and return the index of the value you're looking for (or the
first value that has a particular property).  

\index{search}

For example, the loop in Listing~\ref{lst:vec_search} finds the index of the element 0 in 
{\tt X}:

\begin{lstlisting}[caption={Searching for the last appearance of 0 in a vector)}, label={lst:vec_search}]
for i=1:length(X)
    if X(i) == 0
        ans = i
    end
end
\end{lstlisting}

A funny thing about this loop is that it keeps going after it
finds what it's looking for.  That might be what you want; if the
target value appears more than one, this loop provides the index
of the {\em last} one.

\index{break statement@{\tt break} statement}
\index{statement!{\tt break}}

But if you want the index of the first one (or you know that there
is only one), you can save some unnecessary looping by using the
{\tt break} statement.

\begin{code}
for i=1:length(X)
    if X(i) == 0
        ans = i
        break
    end
end
\end{code}

{\tt break} does pretty much what it sounds like.  It ends the
loop and proceeds immediately to the next statement after the
loop (in this case, there isn't one, so the code ends).
