\chapter*{Introduction}
\adjustmtc
\chapterartfile{book/images/matlab_circleart.eps}

Modeling and simulation are powerful tools for explaining the world, making \linebreak predictions, designing things that work, and making them work better.  Learning to use these tools can be difficult; this book is my attempt to make the experience as enjoyable and productive as possible.

By reading this book---and working on the exercises---you will learn some programming, some modeling, and some simulation.
With basic programming skills, you can create models for a wide range of physical systems.
My goal is to help you develop these skills in a way you can apply immediately to real-world problems.

This book presents the entire modeling process, including model selection, analysis, simulation, and validation.  I explain this process in Chapter~\ref{modeling}, and there are examples throughout the book.

\section{Who This Book Is For}

To make this book accessible to the widest possible audience, I've tried to minimize the prerequisites.

This book is intended for people who have never programmed before.  I start from the beginning, define new terms when they are introduced, and present only the features you need, when you need them.

I assume that you know trigonometry and some calculus, but not very much.  If you understand that a derivative represents a rate of change, that's enough.  You will learn about differential equations and some linear algebra, but I will explain what you need to know as we go along.

I will assume you know basic physics, in particular the concepts of force, acceleration, velocity, and position.  If you know Newton's second law of motion in the form $F = m a$, that's enough.

\section{About This Book}

I have tried to present a small set of tools that provides the most versatility and power, to explain those tools as clearly as possible, and to give you chances to practice what you learn.

Here's what you will find in this book:
\begin{description}
\item [Chapter 1: Modeling and Simulation] Presents the modeling framework we'll use in this book, introduces the MATLAB and Octave programming languages, and helps you debug some of the errors you are likely to make while you are getting started

\item [Chapter 2: Scripts] Introduces scripts, which are files that contain MATLAB/Octave code.  It also presents variables, values, and the assignment statement

\item [Chapter 3: Loops] Presents the \lstinline{for} loop, sequences, series, plotting, and a way of writing programs called incremental development

\item [Chapter 4: Vectors] Introduces vectors, which provide a way to store a sequence of values.  And it presents common vector operators including reduce and apply

\item [Chapter 5: Functions] Discusses name collisions and an important tool for avoiding them: functions.  It also explains input variables and function calls

\item [Chapter 6: Conditionals] Presents conditional statements, which check for conditions and determine the behavior of programs.  And it introduces a program development process called encapsulation and generalization

\item [Chapter 7: Zero-Finding] Introduces \lstinline{fzero}, which is a MATLAB function that finds the zeros, or roots, of nonlinear equations.  It also presents some tips that might help you with debugging

\item [Chapter 8: Functions of Vectors] Combines two topics from previous chapters: vectors and functions.  It presents functions that take vectors as input variables and return them as output variables.  And it \mbox{introduces} logical vectors, which contain a sequence of true and false values

\item [Chapter 9: Ordinary Differential Equations] Introduces the most important idea in the book, differential equations, and two ways to solve them, Euler's method and a MATLAB function called \lstinline{ode45}

\item [Chapter 10: Systems of ODEs] Uses a system of differential equations to simulate the interactions of predator and prey species and presents several ways to plot the results

\item [Chapter 11: Second-Order Systems] Describes Newtonian motion using a second-order differential equation and uses \lstinline{ode45} to simulate falling objects with and without air resistance

\item [Chapter 12: Two Dimensions] Extends the methods from the previous chapter to simulate projectiles like baseballs.  It introduces spatial vectors as a way to represent quantities with two and three dimensions

\item [Chapter 13: Optimization] Introduces \lstinline{fminsearch}, which is a MATLAB function that searches for the maximum or minimum of a function

\item [Chapter 14: Springs and Things] Adds new forces to the toolkit, including spring forces and universal gravitation.  It uses them to simulate the orbit of the Earth around the Sun

\item [Chapter 15: Under the Hood] Reviews some of the MATLAB functions we've used---\lstinline{fzero}, \lstinline{ode45}, and \lstinline{fminsearch}---and explains more about how they work

\end{description}

I hope you enjoy the book and find it valuable.


\section*{Installing Software}

This book is based on MATLAB, a programming language originally \linebreak developed at the University of New Mexico and now produced by MathWorks, Inc.

MATLAB is a high-level language with features that make it well-suited for modeling and simulation, and it comes with a program development environment that makes it well-suited for beginners.

However, one challenge for beginners is that MATLAB uses vectors and matrices for almost everything, which can make it hard to get started.  The organization of this book is meant to help: we start with simple numerical computations, adding vectors in Chapter~\ref{vectors} and matrices in Chapter~\ref{systems}.

Another drawback of MATLAB is that it is ``proprietary''; that is, it belongs to MathWorks, and you can only use it with a license, which can be expensive.

Fortunately, the GNU Project has developed a free, open-source alternative called Octave (see \url{https://www.gnu.org/software/octave}).  

Most programs written in MATLAB can run in Octave without modification, and the other way around.  All programs in this book have been tested with Octave, so if you don't have access to MATLAB, you should be able to work with Octave.  The biggest difference you are likely to see is in the error messages.

To install and run MATLAB, see \url{https://greenteapress.com/matlab/matlab}.

The first time you run it, a start window should appear to guide you through some configuration.

To install Octave, we recommend that you use Anaconda, which is a package management system that makes it easy to work with Octave and \linebreak supporting software.

Anaconda installs everything at the user level, so you can install it without admin or root permissions.  Follow the instructions for your operating system at \url{https://greenteapress.com/matlab/anaconda}.

Once you have Anaconda, you can install Octave by launching the Jupyter Prompt (on Windows) or a Terminal (on Mac OS or Linux), typing the following, and pressing \keycap{enter}:

\begin{code}
***conda install -c conda-forge octave***
\end{code}

Then you can launch it by typing:

\begin{code}
***octave***
\end{code}
and pressing \keycap{enter}.

\section{Working with the Code}

%TODO: Once the title is finalized, I will make these file names consistent

The code for each chapter in this book is in a ZIP file you can download from \url{https://greenteapress.com/matlab/zip}.

Once you have the ZIP file, you can unzip it on the command line by running

\begin{code}
***unzip PhysicalModelingInMatlab-master.zip***
\end{code}

In Windows you can right-click on the ZIP file and select \textbf{Extract All}.

If you open any of these files in MATLAB, you should be able to read the code.  To run it, press the green \textbf{Run} button.

You might get a message like ``File not found in the current folder.''
MATLAB will give you the option to Change Folder or Add to Path.  If you change folders, you will be able to run this file until you change folders again.  If you add to the path, you will always be able to run this file.

However, as you add more folders to the path, you are more likely to run into problems with name collisions (see ``\nameref{collision}'' on page~\pageref{collision}).
I recommend you change folders when necessary and avoid adding folders to the path.


\newpage


\newpage
