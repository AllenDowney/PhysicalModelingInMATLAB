\chapter{Functions of Vectors}

Now that we have functions and vectors, we'll put them together to write functions that take vectors as input variables and return vectors as output variables.  You'll also see two patterns for computing with vectors: existential and universal quantification.


\section{Functions and Vectors}

In this section we'll look at common patterns involving functions and vectors, and you will learn how to write a single function that can work with vectors as well as scalars.

\subsection{Vectors as Input Variables}

Since many of the built-in functions take vectors as arguments,
it should come as no surprise that you can write functions that
take vectors as input.  Here's a simple (but not very useful) example:

\index{vector}
\index{input variable}

\begin{code}
function res = display_vector(X)
    for i=1:length(X)
        display(X(i))
    end
end
\end{code}

There's nothing special about this function.  The only
difference from the scalar functions we've seen is that I used
a capital letter to remind me that {\tt X} is a vector.

\verb"display_vector" doesn't actually return a value; it just displays the elements of the vector it gets as an input variable:

\begin{code}
>> ***display_vector(1:3)***
    1
    2
    3
\end{code}

Here's a more interesting example that encapsulates the code
from Listing~\ref{lst:vec_reduce} (page~\pageref{lst:vec_reduce}) that adds up the elements of a vector:

\index{encapsulation}

\begin{code}
function res = mysum(X)
    total = 0;
    for i=1:length(X)
        total = total + X(i);
    end
    res = total;
end
\end{code}

I called it {\tt mysum} to avoid a collision with the built-in
function {\tt sum}, which does pretty much the same thing.

\index{sum@{\tt sum}}

Here's how you call it from the Command Window:

\begin{code}
>> ***total = mysum(1:3)***
total = 6
\end{code}

Because this function has an output variable, I made a
point of assigning it to a variable.

\index{output variable}


\subsection{Vectors as Output Variables}

There's also nothing wrong with assigning a vector to an output
variable. Here's an example that encapsulates the code from
Listing~\ref{lst:vec_apply} (page~\pageref{lst:vec_apply}):

\begin{code}
function res = mysquare(X)
    for i=1:length(X)
        Y(i) = X(i)^2;
    end
    res = Y;
end
\end{code}

This function squares each element of {\tt X} and stores it as an element of {\tt Y}.  Then it assigns {\tt Y} to the output variable, {\tt res}.  Here's how we use this function:
\index{apply}

\begin{code}
>> ***V = mysquare(1:3)***
V = 1     4     9
\end{code}

The input variable is a vector with the elements {\tt 1,2,3}.  The output variable is a vector with elements {\tt 1,4,9}.




\subsection{Vectorizing Functions}

Functions that work on vectors will almost always work on scalars
as well, because MATLAB considers a scalar to be a vector with
length 1.

\index{vectorizing}
\index{function!vectorizing}

\begin{code}
>> ***mysum(17)***
ans = 17

>> ***mysquare(9)***
ans = 81
\end{code}

Unfortunately, the converse isn't always true.  If you write
a function with scalar inputs in mind, it might not work on vectors.

But it might!  If the operators and functions
you use in the body of your function work on vectors, then your
function will probably work on vectors.

For example, here's the very first function we wrote:

\begin{code}
function res = myfunc(x)
    s = sin(x);
    c = cos(x);
    res = abs(s) + abs(c);
end
\end{code}

And lo!  It turns out to work on vectors:

\begin{code}
>> ***Y = myfunc(1:3)***
Y = 1.3818    1.3254    1.1311
\end{code}

Some of the other functions we wrote don't work on vectors,
but they can be patched up with just a little effort.  For example,
here's {\tt hypotenuse} from Exercise~\ref{hypotenuse_exercise}:

\begin{code}
function res = hypotenuse(a, b)
    res = sqrt(a^2 + b^2);
end
\end{code}

This doesn't work on vectors because the \verb+^+ operator
tries to do matrix exponentiation, which only works on
square matrices.

\index{matrix exponentiation}

\begin{code}
>> ***hypotenuse(1:3, 1:3)***
Error using  ^  (line 51)
Incorrect dimensions for raising a matrix to a power. 
Check that the matrix is square and the power is a scalar. 
To perform elementwise matrix powers, use '.^'.
\end{code}

But if you replace \verb+^+ with the elementwise operator
\verb+.^+, it works!

\index{elementwise operator}

\begin{code}
>> ***A = [3,5,8];***
>> ***B = [4,12,15];***
>> ***C = hypotenuse(A, B)***

C = 5    13    17
\end{code}

The function matches up corresponding elements from the two
input vectors, so the elements of {\tt C} are the hypotenuses of
the pairs $(3,4)$, $(5,12)$, and $(8,15)$, respectively.

In general, if you write a function using only elementwise
operators and functions that work on vectors, the new
function will also work on vectors.


\subsection{Sums and Differences}

Another common vector operation is {\emph cumulative sum}, which takes a vector as an input and computes a new vector that contains all of the partial sums of the original.  In math notation, if $V$ is the original vector, the elements of the cumulative sum, $C$, are:

\index{cumulative sum}
\index{sum!cumulative}

\begin{equation}
C_i = \sum_{j=1}^i V_j
\end{equation}

In other words, the $i$th element of $C$ is the sum of the first
$i$ elements from $V$.  MATLAB provides a function named {\tt cumsum} that computes cumulative sums:

\index{cumsum@{\tt cumsum}}

\begin{code}
>> ***X = 1:5***

X = 1     2     3     4     5

>> ***C = cumsum(X)***

C = 1     3     6    10    15
\end{code}

The inverse operation of {\tt cumsum} is {\tt diff}, which computes
the difference between successive elements of the input vector.

\index{diff@{\tt diff}}

\begin{code}
>> ***D = diff(C)***

D = 2     3     4     5
\end{code}

Notice that the output vector is shorter by one than the input
vector.  As a result, MATLAB's version of {\tt diff} is not
exactly the inverse of {\tt cumsum}.  If it were, we would
expect {\tt cumsum(diff(X)} to be {\tt X}:

\begin{code}
>> ***cumsum(diff(X))***

ans = 1     2     3     4
\end{code}

But it isn't.

\begin{ex}
Write a function named {\tt mydiff} that computes the
inverse of {\tt cumsum}, so that {\tt cumsum(mydiff(X))} and
{\tt mydiff(cumsum(X))} both return {\tt X}.

% mydiff.m
\end{ex}


\subsection{Products and Ratios}

The multiplicative version of {\tt cumsum} is {\tt cumprod},
which computes the {\emph cumulative product}.  In math notation,
that's:

\index{cumulative product}
\index{product!cumulative}
\index{cumprod@{\tt cumprod}}

\begin{equation}
P_i = \prod_{j=1}^i V_j
\end{equation}

In MATLAB, that looks like:

\begin{code}
>> ***V = 1:5***

V = 1     2     3     4     5

>> ***P = cumprod(V)***

P = 1     2     6    24   120
\end{code}

MATLAB doesn't provide the multiplicative version
of {\tt diff}, which would be called {\tt ratio}, and which would
compute the ratio of successive elements of the input vector.

\begin{ex}
Write a function named {\tt myratio} that computes the
inverse of {\tt cumprod}, so that {\tt cumprod(myratio(X))} and
{\tt myratio(cumprod(X))} both
return {\tt X}.

You can use a loop, or if you want to be clever, you can take
advantage of the fact that $e^{\ln a + \ln b} = a b$.

If you apply {\tt myratio} to a vector that contains Fibonacci
numbers, you can confirm that the ratio of successive elements
converges on the golden ratio, $(1+\sqrt{5})/2$ (see
Exercise~\ref{fibratio}).

% TODO
\end{ex}

\section{Computing with Vectors}

In this section we'll look at two common patterns for working with vectors and connect them to the corresponding ideas from mathematics, existential and universal quantification.  And you'll learn about logical vectors, which contain the Boolean values 0 and 1. 

\subsection{Existential Quantification}

\index{existential quantification}
\index{quantification!existential}

It's often useful to check the elements of a vector to see if there
are any that satisfy a condition.  For example, you might want to
know if there are any positive elements.  In mathematical terms, checking whether something exists is called is called {\emph existential quantification}, and it's denoted with
the symbol $\exists$, which is pronounced ``there exists.''  For example,
this expression
%
\[ \exists x \mbox{~in~} S: x>0  \]
%
means, ``there exists some element $x$ in the set $S$ such that
$x>0$.''  In MATLAB it's natural to express this idea with a logical
function, like {\tt exists}, that returns 1 if there is such an
element and 0 if there is not.

\begin{code}
function res = exists(X)
    for i=1:length(X)
        if X(i) > 0
            res = 1;
(*\codewingding{1}*)            return
        end
    end
    res = 0;
end
\end{code}

We haven't seen the {\tt return} statement before \wingding{1}; it's similar
to {\tt break} except that it breaks out of the whole function, not
just the loop.  That behavior is what we want here because as soon
as we find a positive element, we know the answer (it exists!) and
we can end the function immediately without looking at the rest
of the elements.

\index{return statement@{\tt return} statement}
\index{statement!{\tt return}}

If we get to the end of the loop, that means we didn't find what
we were looking for, so the result is 0.

\subsection{Universal Quantification}

\index{universal quantification}
\index{quantification!universal}

Another common operation on vectors is to check whether {\em all}
of the elements satisfy a condition, which is called {\emph universal quantification}, denoted with
the symbol $\forall$, and pronounced ``for all.''  So this
expression
%
\[ \forall x \mbox{~in~} S: x>0 \]
%
means ``for all elements, $x$, in the set $S$, $x>0$.''

One way evaluate this expression in MATLAB is to reduce the problem to
existential quantification; that is, to rewrite

\begin{equation}
\forall x \mbox{~in~} S: x>0
\end{equation}

as

\begin{equation}
\sim \exists x \mbox{~in~} S: x \le 0
\end{equation}

Where $\sim \exists$ means ``does not exist.''
In other words, checking that all the elements are positive is
the same as checking that there are no elements
that are non-positive.

\begin{ex}
Write a function named {\tt forall} that
takes a vector and returns 1 if all of the elements are positive
and 0 if there are any non-positive elements.
\end{ex}




\subsection{Logical Vectors}

When you apply a logical operator to a vector, the result is a 
{\emph logical vector}: a vector whose elements are the logical
values 1 and 0. Let's look at an example:

\index{logical vector}
\index{vector!logical}

\begin{code}
>> ***V = -3:3***

V = -3    -2    -1     0     1     2     3

>> ***L = V>0***

L =  0     0     0     0     1     1     1
\end{code}

In this example, {\tt L} is a logical vector whose elements
correspond to the elements of {\tt V}.  For each positive element of
{\tt V}, the corresponding element of {\tt L} is 1.

Logical vectors can be used like flags to store the state of
a condition.  And they are often used with the {\tt find} function,
which takes a logical vector and returns a vector that contains
the indices of the elements that are ``true''.

\index{flag}
\index{find@{\tt find}}

Applying {\tt find} to {\tt L} from the example above yields

\begin{code}
>> ***find(L)***

ans = 5     6     7
\end{code}

which indicates that elements 5, 6 and 7 have the value 1.

If there are no ``true'' elements, the result is an empty vector.

\begin{code}
>> ***find(V>10)***

ans = Empty matrix: 1x0
\end{code}

This example computes the logical vector and passes it as an
argument to {\tt find} without assigning it to an intermediate
variable.  You can read this version abstractly as ``find
the indices of elements of {\tt V} that are greater than 10.''

We can also use {\tt find} to write {\tt exists} more concisely:

\begin{code}
function res = exists(X)
    L = find(X>0)
    res = length(L) > 0
end
\end{code}

\begin{ex}
Write a version of {\tt forall} using {\tt find}.
\end{ex}


\section{Debugging in Four Acts}

\index{debugging}
\index{reading}
\index{running}
\index{ruminating}
\index{retreating}

When you're debugging a program, and especially if you're working on a hard bug, there are four things to try:

\begin{description}

\item[reading:] Examine your code, read it back to yourself, and
check that it means what you meant to say.

\item[running:] Experiment by making changes and running different
versions.  Often if you display the right thing at the right place
in the program, the problem becomes obvious, but you might have to invest time building scaffolding.

\item[ruminating:] Take some time to think!  What kind of error
is it: syntax, run-time, or logical?  What information can you get from
the error messages, or from the output of the program?  What kind of
error could cause the problem you're seeing?  What did you change
last, before the problem appeared?

\item[retreating:] At some point, the best thing to do is back
off, undoing recent changes, until you get back to a program that
works, and that you understand.  Then you can starting rebuilding.

\end{description}

Beginning programmers sometimes get stuck on one of these activities
and forget the others.  Each activity comes with its own failure
mode.

For example, reading your code might help if the problem is a
typographical error, but not if the problem is a conceptual
misunderstanding.  If you don't understand what your program does, you
can read it 100 times and never see the error, because the error is in
your head.

Running experiments can help, especially if you run small, simple
tests.  But if you run experiments without thinking or reading your
code, you might fall into a pattern I call ``random walk programming,''
which is the process of making random changes until the program
does the right thing.  Needless to say, random walk programming
can take a long time.

\index{random walk programming}

The way out is to take more time to think.  Debugging is like an
experimental science.  You should have at least one hypothesis about
what the problem is.  If there are two or more possibilities, try to
think of a test that would eliminate one of them.

\index{hypothesis}

Taking a break sometimes helps with the thinking.  So does talking.
If you explain the problem to someone else (or even yourself), you
will sometimes find the answer before you finish asking the question.

But even the best debugging techniques will fail if there are too many
errors, or if the code you are trying to fix is too big and
complicated.  Sometimes the best option is to retreat, simplifying the
program until you get to something that works, and then rebuild.

Beginning programmers are often reluctant to retreat, because
they can't stand to delete a line of code (even if it's wrong).
If it makes you feel better, copy your program into another file
before you start stripping it down.  Then you can paste the pieces
back in a little bit at a time.

\index{debugging!Eighth Theorem}

To summarize, here's the Eighth Theorem of Debugging:

\begin{quote}
Finding a hard bug requires reading, running, ruminating,
and sometimes retreating.  If you get stuck on one of these
activities, try the others.
\end{quote}

\section{Summary}

This chapter presents patterns for working with vectors and functions.  At this point, you know how write functions that take vectors as input variables and return vectors as output variables.

Some of the functions in this chapter are not idiomatic MATLAB; many of them can be done more simply using built-in MATLAB operators and functions rather than writing them yourself.  These simple examples are meant to demonstrate concepts you will need to know when you work on more complicated examples.

In the next chapter we will apply to tools we have learned so far to the central goal of this book, modeling physical systems.


\section{Glossary}

\begin{description}

\item[existential quantification:] A logical condition that expresses the idea that ``there exists'' an element of a set with a certain property.

\item[universal quantification:] A logical condition that expresses
the idea that all elements of a set have a certain property.

\item[logical vector:] A vector, usually the result of applying a logical operator to a vector, that contains logical values 1 and 0.

\end{description}

%\section{Exercises}

%\begin{ex}
%\end{ex}

%TODO: any exercises for this chapter?