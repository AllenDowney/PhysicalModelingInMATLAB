\chapter{Two Dimensions}

In the previous chapter we solved a one-dimensional problem, a penny falling from the Empire State Building.  Now we'll solve a two-dimensional problem, finding the trajectory of a baseball.

\index{baseball}

To do that, we'll use spatial vectors to represent quantities in two and three dimensions, including force, acceleration, velocity, and position.

\section{Spatial Vectors}
\label{spacial}

The word {\em vector} means different
things to different people.  In MATLAB, a vector is a matrix that has
either one row or one column.  So far, we've used MATLAB vectors to
represent:

\begin{description}

\item[Sequences] A mathematical sequence, like the Fibonacci numbers, is a set of values identified by integer indices; in Chapter~\ref{vecseq}, we used a MATLAB vector to store the elements of a sequence. 

\item[State vectors] A state vector is a set of values that
describes the state of a physical system.  When you call
\lstinline{ode45}, you give it initial conditions in a state
vector.  Then, when \lstinline{ode45} calls your rate function, it
gives you a state vector.

\item[Time series] One of the results from \lstinline{ode45} is a vector that represents a sequence of time values.

\end{description}

\index{vector!spatial}
\index{spatial vector}
\index{sequence}
\index{state}
\index{time series}

In this chapter we'll see another use of MATLAB vectors: representing
\emph{spatial vectors}.  A spatial vector represents a multidimensional physical quantity like position, velocity, acceleration, or force.

\index{position}

For example, to represent a position in two-dimensional space, we can use a vector with two elements:

\begin{code}
>> ***P = [3 4]***
\end{code}

To interpret this vector, we have to know the coordinate system it is defined in.  Most commonly, we use a Cartesian system where the x-axis points east and the y-axis points north.  In that case \lstinline{P} represents a point 3~units east and 4~units north of the origin.

\index{Cartesian coordinates}
\index{magnitude}
\index{direction}
\index{Pythagorean theorem}

When a spatial vector is represented in this way, we can use it to compute the magnitude and direction of a physical quantity.  
For example, the {\em magnitude} of \lstinline{P} is the distance from the origin to \lstinline{P}, which is the hypotenuse of the triangle with sides \lstinline{P(1)} and \lstinline{P(2)}.  
We can compute it using the Pythagorean theorem:

\begin{code}
>> ***sqrt(P(1)^2 + P(2)^2)***
ans = 5
\end{code}

Or we can do the same thing using the function \lstinline{norm}, which computes the
``Euclidean norm'' of a vector, which is its magnitude:

\index{norm@\lstinline{norm}}
\index{Euclidean norm}

\begin{code}
>> ***norm(P)***
ans = 5
\end{code}

There are two ways to get the {\em direction} of a vector.  One convention is to compute the angle between the vector and the x-axis:

\begin{code}
>> ***atan2(P(2), P(1))***
ans = 0.9273
\end{code}

In this example, the angle is about \SI{0.9}{\radian}.  But for computational purposes, we often represent direction with a \emph{unit vector}, which is a vector with length~1.  To get a unit vector we can divide a vector by its length:

\begin{code}
function res = hat(V)
    res = V / norm(V)
end
\end{code}
 
This function takes a vector, \lstinline{V}, and returns a unit vector with the same direction as \lstinline{V}.  It's called \lstinline{hat} because in mathematical notation, unit vectors are written with a ``hat'' symbol.  
For example, the unit vector with the same direction as $\vec{P}$ would be written $\uvec{P}$. 

\index{unit vector}
\index{vector!unit}

\section{Adding Vectors}

Vectors are useful for representing quantities like force and acceleration because we can add them up without having to think explicitly about direction.

\index{force}
\index{acceleration}

As an example, suppose we have two vectors representing forces:

\begin{code}
>> ***A = [2, 4];***
>> ***B = [2, -2];***
\end{code}

\lstinline{A} represents a force pulling northeast; \lstinline{B} represents a force pulling southeast, as shown in Figure~\ref{fig:vector2}

\begin{figure}[ht]
\centerline{\includegraphics[height=3in]{book/figs/vector2.eps}}
\caption{The sum of two forces represented by vectors}
\label{fig:vector2}
\end{figure}

To compute the sum of these forces, all we have to do is add the vectors:

\begin{code}
>> ***A + B***
ans = 4     2
\end{code}

Later in the chapter we'll use vector addition to add accelerations due to different forces acting on a baseball.


\section{ODEs in Two Dimensions}
\label{projectile}

So far we've used \lstinline{ode45} to solve a system of first-order equations and a single second-order equation.  Now we'll take one more step, solving a system of second-order equations.

As an example, we'll simulate the flight of a baseball.
If there is no wind and no spin on the ball, the ball travels in a vertical plane, so we can think of the system as two-dimensional, with $x$ representing the horizontal distance
traveled from the starting place and $y$ representing height or altitude.

\index{baseball}
\index{altitude}
\index{rate function}

Listing~\ref{lst:baseball_rate_func} shows a rate function we can use to simulate this system with \lstinline{ode45}:

\begin{lstlisting}[caption={A rate function we can use to model the flight of a baseball}, label={lst:baseball_rate_func}]
function res = rate_func(t, W)
    P = W(1:2);
    V = W(3:4);

    dPdt = V;
    dVdt = acceleration(t, P, V);

    res = [dPdt; dVdt];
end

function res = acceleration(t, P, V)
    g = 9.8;             % acceleration due to gravity in m/s^2
    a_gravity = [0; -g];
    res = a_gravity;
end
\end{lstlisting}

The second argument of \lstinline{rate_func} is understood to be a vector,
\lstinline{W}, with four elements.  The first two are assigned to \lstinline{P},
which represents position; the last two are assigned to \lstinline{V}, which
represents velocity. Both \lstinline{P} and \lstinline{V} have two elements,
representing the $x$ and $y$ components.

\index{derivative}
\index{velocity}

The goal of the rate function is to compute the derivative of \lstinline{W}, so the output has to be a vector with four elements, where the first two represent the derivative of \lstinline{P}  and the last two represent the derivative of \lstinline{V}.
The derivative of \lstinline{P} is velocity.  We don't have to compute it; we were given it as part of \lstinline{W}.
The derivative of \lstinline{V} is acceleration.  To compute it, we call \lstinline{acceleration}, which takes as input variables time, position, and velocity.  In this example we don't use any of the input variables, but we will soon.

\index{air resistance}
\index{gravity}

For now we'll ignore air resistance, so the only force on the baseball is gravity.  We represent acceleration due to gravity with a vector that has magnitude \lstinline{g} and direction along the negative y-axis.

Let's assume that a ball is batted from an initial position \SI{1}{\meter} above the home plate, with an initial velocity of \SI{40}{\meter\per\second} in the horizontal and \SI{30}{\meter\per\second} in the vertical direction.

\index{initial condition}

Here's how we can call \lstinline{ode45} with these initial conditions:
 
\begin{code}
    P = [0; 1];       % initial position in m
    V = [40; 30];     % initial velocity in m/s
    W = [P; V];       % initial condition
    
    tspan = [0 8]
    [T, M] = ode45(@rate_func, tspan, W);
\end{code}

\lstinline{P} and \lstinline{V} are column vectors because we put semicolons between the elements.  
So \lstinline{W} is a column vector with four elements.
\lstinline{tspan} specifies that we want to run the simulation for \SI{8}{\second}.

\index{time span}
\index{output variable}

The output variables from \lstinline{ode45} are a vector, 
\lstinline{T}, that contains time values, and a matrix, \lstinline{M}, with four columns: the first two are position; the last two are velocity.

Here's how we can plot position as a function of time:

\begin{code}
    X = M(:, 1);
    Y = M(:, 2);
    
    plot(T, X)
    plot(T, Y)
\end{code}

\lstinline{X} and \lstinline{Y} get the first and second columns from \lstinline{M}, which are the $x$ and $y$ position coordinates.

\begin{figure}[ht]
\centerline{\includegraphics[height=3in]{book/figs/baseball1.eps}}
\caption{Simulated flight of a baseball neglecting drag force}
\label{fig:baseball1}
\end{figure}

Figure~\ref{fig:baseball1} shows what these coordinates look like as a function of time.  The x-coordinate increases linearly because the $x$ velocity is constant.  The y-coordinate goes up and down, as we expect.

The simulation ends just before the ball lands, having traveled almost \SI{250}{\meter}.  That's substantially farther than a real baseball would travel, because we have ignored air resistance, or ``drag force.''


\section{Drag Force}
\label{drag}

\index{drag}
\index{force!drag}

A simple model for the drag force on a baseball is:

\begin{equation*}
    \vec{F}_\mathrm{d} = - \frac{1}{2} \, \rho v C_\mathrm{d} A \uvec{V},
\end{equation*}
where $\vec{F}_\mathrm{d}$ is a vector that represents the force on the baseball
due to drag, 
$\rho$ is the density of air, 
$C_\mathrm{d}$ is the drag coefficient, and
$A$ is the cross-sectional area.

\index{drag coefficient}
\index{coefficient!drag}

$\vec{V}$ is the baseball's velocity vector; $v$ is the magnitude of $\vec{V}$ and $\uvec{V}$ is a unit vector in the same direction as $\vec{V}$.  The minus sign at the beginning means that the result is in the opposite direction to $\vec{V}$.

\index{unit vector}

The function in Listing~\ref{lst:baseball_drag} computes the drag force on a baseball:

\begin{lstlisting}[caption={A function that calculates the drag force on a baseball}, label={lst:baseball_drag}]
 function res = drag_force(V)
    C_d = 0.3;      % dimensionless
    rho = 1.3;      % kg / m^3
    A = 0.0042;     % m^2
    v = norm(V);    % m/s

    res = -1/2 * C_d * rho * A * v * V;
end
\end{lstlisting}
  
The drag coefficient for a baseball is about 0.3.  
The density of air at sea level is about \SI{1.3}{\kilogram\per\meter\cubed}.
The cross-sectional area of a baseball is \SI{0.0042}{\meter\squared}.

\index{density}
\index{cross-sectional area}

Now we have to update \lstinline{acceleration} to take drag into account:

\begin{code}
function res = acceleration(t, P, V)
    g = 9.8;                       % acceleration due to gravity in m/s^2
    a_gravity = [0; -g];

    m = 0.145;                     % mass in kilograms
    a_drag = drag_force(V) / m;
    res = a_gravity + a_drag;
end
\end{code}

As in Listing~\ref{lst:baseball_rate_func}, \lstinline{acceleration} represents acceleration due to gravity with a vector that has magnitude \lstinline{g} and direction along the negative y-axis.
But now it also computes drag force and divides by the mass of the baseball to get acceleration due to drag.
Finally, it adds \lstinline{a_gravity} and \lstinline{a_drag} to get the total acceleration of the baseball.

\begin{figure}[ht]
\centerline{\includegraphics[height=3in]{book/figs/vector3.eps}}
\caption{Diagram of velocity, $\vec{V}$, acceleration due to drag force, 
$\vec{D}$, acceleration due to gravity, $\vec{G}$, and total acceleration, $\vec{A}$}
\label{fig:vector3}
\end{figure}

Figure~\ref{fig:vector3} shows the following quantities graphically:  (1) acceleration due to drag, $\vec{D}$, which is in the opposite direction to (2)~velocity, $\vec{V}$; (3)~acceleration due to gravity, $\vec{G}$, is straight down; and (4)~total acceleration, $\vec{A}$, is the sum of $\vec{D}$ and $\vec{G}$.

\index{acceleration}

\begin{figure}
\centerline{\includegraphics[height=3in]{book/figs/baseball2.eps}}
\caption{Simulated flight of a baseball including drag force}
\label{fig:baseball2}
\end{figure}

Figure~\ref{fig:baseball2} shows the results from \lstinline{ode45}.  The ball lands after about \SI{5}{\second}, having traveled less than \SI{150}{\meter}, substantially less than what we got without air resistance, about \SI{250}{\meter}.

This result suggests that ignoring air resistance is not a good choice for modeling a baseball.


\section{What Could Go Wrong?}

What could go wrong?  Well, \lstinline{vertcat} for one.  To explain
what that means, I'll start with \emph{concatenation}, which is
the operation of joining two matrices into a larger matrix.
``Vertical concatenation'' joins the matrices by stacking them on
top of each other; ``horizontal concatenation'' lays them
side by side.

\index{concatenation}
\index{vertical concatenation}
\index{vertcat@\lstinline{vertcat}}

Here's an example of horizontal concatenation with row vectors:

\begin{code}
>> ***x = 1:3***
x = 1     2     3

>> ***y = 4:5***
y = 4     5

>> ***z = [x, y]***
z = 1     2     3     4     5
\end{code}

Inside brackets, the comma operator performs horizontal concatenation.

The vertical concatenation operator is the semicolon.  Here's an
example with matrices:

\index{comma operator}
\index{operator!comma}

\begin{code}
>> ***X = zeros(2, 3)***

X =  0     0     0
     0     0     0

>> ***Y = ones(2, 3)***

Y =  1     1     1
     1     1     1

>> ***Z = [X; Y]***

Z =  0     0     0
     0     0     0
     1     1     1
     1     1     1
\end{code}

These operations only work if the matrices are the same size along
the dimension where they are glued together.  If not, you get:

\begin{code}
>> ***a = 1:3***

a = 1     2     3

>> ***b = a'***

b =  1
     2
     3

>> ***c = [a, b]***
Error using horzcat
Dimensions of matrices being concatenated are not consistent.

>> ***c = [a; b]***
Error using vertcat
Dimensions of matrices being concatenated are not consistent.
\end{code}

In this example, \lstinline{a} is a row vector and \lstinline{b} is a column
vector, so they can't be concatenated in either direction.

Reading the error messages, you might guess that \lstinline{horzcat}
is the function that performs horizontal concatenation, and likewise
with \lstinline{vertcat} and vertical concatenation.  You would be correct.

\index{horzcat@\lstinline{horzcat}}

In Listing~\ref{lst:baseball_rate_func} we used vertical concatenation to pack \lstinline{dPdt} and \lstinline{dVdt} into the output variable:

\begin{code}
function res = rate_func(t, W)
    P = W(1:2);
    V = W(3:4);

    dPdt = V;
    dVdt = acceleration(t, P, V);

    res = [dPdt; dVdt];
end
\end{code}

As long as \lstinline{dPdt} and \lstinline{dVdt} are column vectors,
the semicolon performs vertical concatenation, and the result is
a column vector with four elements.  But if either of them is a
row vector, that's trouble.

\index{column vector}

\lstinline{ode45} expects the result from \lstinline{rate_func} to be a
column vector, so if you are working with \lstinline{ode45}, it's
probably a good idea to make {\em everything} a column vector.

In general, if you run into problems with \lstinline{horzcat} and
\lstinline{vertcat}, use \lstinline{size} to display the dimensions of the operands,
and make sure you are clear on which way your vectors go.

\section{Chapter Review}

In this chapter we simulated the flight of a baseball with and without air resistance, and saw that the difference is substantial.
We can conclude that it's important to model air resistance if we want to make accurate predictions about baseballs and similar projectiles.

Here are some of the terms from this chapter you might want to remember.

A {\em spatial vector} is a value that represents a multidimensional physical quantity like position, velocity, acceleration, or force.
A spatial vector has a direction and a magnitude.  The magnitude is also called the {\em norm} of the vector.
A {\em unit vector} is a vector with norm 1, which is often used to represent a
direction.

{\em Concatenation} is the operation of joining two vectors or matrices end-to-end to
form a new vector or matrix.

In the next chapter we'll continue with the baseball example, using \lstinline{fzero}, which we saw in Chapter~\ref{fzero}, and a new tool for optimization, called \lstinline{fminsearch}.  We'll also see a simple way to animate the solution of a differential equation.


\section{Exercises}

Before you go on, you might like to work on the following exercises.

\subsection{Exercise 1}

\index{Boston Red Sox}
\index{density}

When the Boston Red Sox won the World Series in 2007, they played the
Colorado Rockies at their home field in Denver, Colorado.  Find an
estimate of the density of air in the Mile High City.  What effect
does this have on drag?  What effect does it have on the distance the baseball travels?

\subsection{Exercise 2}

\index{drag}
\index{air resistance}
\index{coefficient!drag}

The actual drag on a baseball is more complicated than what is
captured by our simple model.  In particular, the drag coefficient
depends on velocity.  You can get some of the details from Robert~K.\ Adair's {\em The
Physics of Baseball} (Harper Perennial, 2002); the figure you need is reproduced at \url{https://greenteapress.com/matlab/drag}.

Use this data to specify a more realistic model of drag and modify your
program to implement it.  How big is the effect on the distance the baseball travels?


\subsection{Exercise 3}
\label{cannon}

\index{cannon}
\index{human cannonball}

According to Wikipedia, the record distance for a human cannonball is 59.05 meters (see \url{https://greenteapress.com/matlab/cannon}).

Modify the example from this chapter to simulate the flight of a human cannonball.  You might have to do some research to find the drag coefficient and cross sectional area for a flying human.

Find the initial velocity (both magnitude and direction) you would need to break this record.  You might have to experiment to find the optimal launch angle.

How much acceleration can a human withstand without injury?  At this maximum acceleration, how long would the barrel of the cannon have to be to reach the initial velocity needed to break the record?



