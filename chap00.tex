\chapter{Preface}


Modeling and simulation are powerful tools for explaining the world, making predictions, designing things that work, and making them work better.  Learning to use these tools can be difficult; this book is my attempt to make the experience as enjoyable and productive as possible.

By reading this book --- and working on the exercises --- you will learn some programming, some modeling, and some simulation:

\begin{itemize}

\item With basic programming skills, you can create models for a wide range of physical systems.
My goal is to help you develop these skills in a way you can apply immediately to real-world problems.

\item This book presents the entire modeling process, including model selection, analysis, simulation, and validation.  I explain this process in Chapter~\ref{modeling}, and there are examples throughout the book.

\item Simulation is an approach to modeling that uses computer programs to  implement models and generate predictions.  This book shows how simulations are used to run experiments, answer questions, and guide decision-making.

\end{itemize}

To make this book accessible to the widest possible audience, I try to minimize the ``prerequisites''.  

In particular, this book is intended for people who have never programmed before.  I start from the beginning, define new terms when they are introduced, and present only the features you need, when you need them.

I assume that you know trigonometry and some calculus, but not much.  If you understand that a derivative represents a rate of change, that's enough.  You will learn about differential equations and some linear algebra, but I will explain what you need to know as we go along.

I assume you know basic physics, in particular the concepts of force, acceleration, velocity, and position.  If you know Newton's second law of motion in the form $F = m a$, that's enough.

You will learn to use numerical methods to search for roots of non-linear equations, to solve differential equations, and to search for optimal solutions.  You will learn how to use these methods first; then in Chapter~\ref{how} you will learn more about how they work.  But if you can't stand the suspense, you can look ``under the hood'' whenever you want.

I have tried to present a small set of tools that provides the most versatility and power, to explain them as clearly as possible, and to give you chances to practice what you learn.

I hope you enjoy the book and find it valuable.


\section*{Installing software}

This book is based on MATLAB, a programming language originally developed at the University of New Mexico and now produced by MathWorks, Inc.  

MATLAB is a high-level language with features that make it well-suited for modeling and simulation, and it comes with a program development environment that makes it well-suited for beginners.

However, one challenge for beginners is that MATLAB uses vectors and matrices for almost everything, which can make it hard to get started.  The organization of this book is meant to help: we start with simple numerical computations, adding vectors in Chapter~\ref{vectors} and matrices in Chapter~\ref{systems}.

Another drawback of MATLAB is that it is ``proprietary''; that is, it belongs to MathWorks, and you can only use it with a license, which can be expensive.

Fortunately, the GNU Project has developed a free, open-source alternative called Octave (see \url{https://www.gnu.org/software/octave/}).  

Most programs written in MATLAB can run in Octave without modification, and the other way around.  All programs in this book have been tested with Octave, so if you don't have access to MATLAB, you should be able to work with Octave.  The biggest difference you are likely to see is in the error messages.

To install and run MATLAB, see \url{https://www.mathworks.com/downloads/web_downloads/}.

To install Octave, we strongly recommend that you use Anaconda, which is a package management system that makes it easy to work with Octave and supporting software.

Anaconda installs everything at the user level, so you can install it without admin or root permissions.  Follow the instructions for your operating system at \url{https://www.anaconda.com/download}.

Once you have Anaconda, you can install Octave by launching the Jupyter Prompt (on Windows) or a Terminal (on Mac OS or Linux) and running:

\begin{code}
conda install -c conda-forge octave
\end{code}

Then you can launch it from the command line like this:

\begin{code}
octave
\end{code}

The first time you run it, a start window should appear to guide you through some configuration.


\section{Working with the code}

%TODO: apply link shorteners

The code for each chapter in this book is in a Zip file you can download from \url{https://github.com/AllenDowney/ModSimMatlab/raw/master/ModSimMatlabCode.zip}.

Once you have the Zip file, you can unzip it on the command line by running

\begin{code}
unzip ModSimMatlabCode.zip
\end{code}

In Windows you can right-click on the Zip file and select {\sf Extract All}.

If you open any of these files in MATLAB, you should be able to read the code.  To run it, press the green {\sf Run} button.

You might get a message like, ``File not found in the current folder''.
MATLAB will give you the option to {\sf Change Folder} or {\sf Add to Path}.  If you change folders, you will be able to run this file until you change folder again.  If you add to the path, you will always be able to run this file.

However, as you add more folders to the path, you are more likely to run into problems with name collisions (see Section~\ref{collision}).  I recommend you change folders when necessary and avoid adding folders to the path.


\newpage

\section*{Contributor's list}

If you have suggestions and corrections, please send them to:\\
\verb"mod_sim_matlab@greenteapress.com".


People who have found errors and helped improve this book include
Michael Lintz, 
Kaelyn Stadtmueller, 
Roydan Ongie, 
Keerthik Omanakuttan, 
Pietro Peterlongo, 
Li Tao, 
Steven Zhang, 
Elena Oleynikova, 
Kelsey Breseman, 
Philip Loh, 
Harold Jaffe, 
Vidie Pong, 
Nik Martelaro, 
Arjun Plakkat, 
Zhen Gang Xiao, 
Zavier Patrick Aguila, 
Michael Cline,
Craig Scratchley,
Matt Wiens,
Denny Chen.

\newpage
