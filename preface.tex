% Preface

\chapter*{Preface}

Most books that use Octave are aimed at readers who know how
to program. This book is for people who have never programmed
before.

As a result, the order of presentation is unusual. The book starts
with scalar values and works up to vectors and matrices very
gradually. This approach is good for beginning programmers, because
it is hard to understand composite objects until you understand basic
programming semantics. But there are problems:

\begin{itemize}

\item The Octave documentation is written in terms of matrices,
and so are the error messages.
To mitigate this problem, the book explains the necessary
vocabulary early and deciphers some of the messages that
beginners find confusing.

\item Many of the examples in the first half of the book are
not idiomatic Octave. I address this problem in the second
half by translating the examples into a more standard style.

\end{itemize}

The book puts a lot of emphasis on functions, in part because they are
an important mechanism for controlling program complexity, and also
because they are useful for working with Octave tools like {\tt fzero}
and {\tt ode45}.

I assume that readers know calculus, differential equations, and
physics, but not linear algebra. I explain the math as I go along,
but the descriptions might not be enough for someone who hasn't seen
the material before.

There are small exercises within each chapter, and a few larger
exercises at the end of some chapters.

This is Version 1.1 of the book. I used Version 0.9 for my class
in Fall 2007. I made some corrections based on feedback from students
and others, then added some exercises and additional chapters.
If you have suggestions and corrections, please send them to
{\tt downey@allendowney.com}.

\noindent Allen B. Downey \\
\noindent Needham, MA \\
\noindent December 28, 2007

\vspace{0.1in}

\section*{Contributor's list}

The following are some of the people who have contributed to this
book:

\begin{itemize}

\item Michael Lintz spotted the first (of many) typos.

\item Kaelyn Stadtmueller reminded me of the importance of linking
verbs.

\item Roydan Ongie knows a matrix when he sees one (and caught a typo).

\item Keerthik Omanakuttan knows that acceleration is not the
second derivative of acceleration.

\item Pietro Peterlongo pointed out that Binet's formula is an
exact expression for the $n$th Fibonacci number, not an approximation.

\item Li Tao pointed out several errors.

\item Steven Zhang pointed out an error and a point of confusion
in Chapter 11.

\item Elena Oleynikova pointed out the ``gotcha'' that script file names
can't have spaces.

\item Kelsey Breseman pointed out that numbers as footnote markers
can be confused with exponents, so now I am using symbols.

\item Philip Loh sent me some updates for recent revisions of Octave.

\item Harold Jaffe spotted a typo.

\item Vidie Pong pointed out the problem with spaces in filenames.

\end{itemize}

% ENDCONTRIB