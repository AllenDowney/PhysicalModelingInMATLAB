\chapter{Functions}
\label{functions}

This chapter introduces the most important idea in computer programming: functions! 
To explain why functions are so important, I'll start by explaining one of the problems they solve: name collisions.

\index{function}

\section{Name Collisions}
\label{collision}

\index{name collision}
\index{collision!name}
\index{workspace}

All scripts run in the same workspace, so if one script changes the value of a variable, all other scripts see the change.  With a small number of simple scripts that's not a problem, but eventually the interactions between scripts become unmanageable.

For example, the following script computes the
sum of the first \lstinline{n} terms in a geometric sequence, but it also
has the {\em side effect} of assigning values to \lstinline{A1}, \lstinline{total},
\lstinline{i}, and \lstinline{a}.

\begin{code}
A1 = 1;
total = 0;
for i=1:10
    a = A1 * 0.5^(i-1);
    total = total + a;
end
ans = total
\end{code}

If you were using any of those variable names before calling this
script, you might be surprised to find, after running the script,
that their values had changed.  If you have two scripts that use
the same variable names, you might find that they work separately
and then break when you try to combine them.  This kind of
interaction is called a \emph{name collision}.

\index{script}

As the number of scripts you write increases, and they get longer
and more complex, name collisions become more of a problem.  Avoiding
this problem is one of several motivations for functions.

\section{Defining Functions}

A {\em function} is like a script, except that each function has its own workspace, so any variables defined
inside a function only exist while the function is running, and don't
interfere with variables in other workspaces, even if they have the
same name. 
Function inputs and outputs are defined carefully to avoid
unexpected interactions.

To define a new function, you create an M-file with the name you
want, and put a function definition in it.  For example, to create
a function named {\tt myfunc}, create an M-file named {\em myfunc.m}
and put the following definition into it:

\index{M-file}
\index{script}
\index{function definition}

\begin{lstlisting}[caption={A function definition}, label={lst:function_def}]
function res = myfunc(x)
    s = sin(x)
    c = cos(x)
    res = abs(s) + abs(c)
end
\end{lstlisting}

The first non-comment word of the file has to be \lstinline{function}, because
that's how MATLAB tells the difference between a script and a function
file.

\index{compound statement}
\index{definition!function}

A function definition is a compound statement.  The first line
is called the {\em signature} of the function; it defines
the inputs and outputs of the function.  In Listing~\ref{lst:function_def} the {\em input variable} is named \lstinline{x}.  When this function is called, the
argument provided by the user will be assigned to \lstinline{x}.

\index{input variable}
\index{variable!input}

\index{output variable}
\index{variable!output}

The {\em output variable} is named \lstinline{res}, which is short for
``result.''  You can call the output variable whatever you want, but
as a convention, I like to call it \lstinline{res}.  Usually the last
thing a function does is assign a value to the output variable.

\index{res@\lstinline{res}}
\index{ans@\lstinline{ans}}

Once you've defined a new function, you call it the same way you
call built-in MATLAB functions.  If you call the function as a statement,
MATLAB puts the result into \lstinline{ans}:

\begin{code}
>> ***myfunc(1)***

s = 0.84147098480790

c = 0.54030230586814

res = 1.38177329067604

ans = 1.38177329067604
\end{code}

But it's more common (and better style) to assign the result to
a variable:

\begin{code}
>> ***y = myfunc(1)***

s = 0.84147098480790

c = 0.54030230586814

res = 1.38177329067604

y = 1.38177329067604
\end{code}

While you're debugging a new function, you might want to display
intermediate results like this, but once it's working, you'll want
to add semicolons to make it a {\em silent function}.  A silent function
computes a result but doesn't display
anything (except sometimes warning messages). Most built-in
functions are silent.

\index{silent function}
\index{function!silent}
\index{workspace}

Each function has its own workspace, which is created when the
function starts and destroyed when the function ends.  If you try to
access (read or write) the variables defined inside a function, you
will find that they don't exist.

\begin{code}
>> ***clear***
>> ***y = myfunc(1);***
>> ***who***
Your variables are: y

>> ***s***
Undefined function or variable 's'.
\end{code}

The only value from the function that you can access is the result,
which in this case is assigned to \lstinline{y}.

If you have variables named \lstinline{s} or \lstinline{c} in your workspace
before you call \lstinline{myfunc}, they will still be there when the
function completes.

\begin{code}
>> ***s = 1;***
>> ***c = 1;***
>> ***y = myfunc(1);***
>> ***s, c***

s = 1
c = 1
\end{code}

So inside a function you can use whatever variable names you
want without worrying about collisions.

\index{name collision}
\index{collision!name}


\section{Function Documentation}

% \index{documentation}
\index{comment}

At the beginning of every function file, you should include a comment
that explains what the function does:

\index{documentation!function}

\begin{code}
% res = myfunc(x)
% Compute the Manhattan distance from the origin to the
% point on the unit circle with angle (x) in radians.

function res = myfunc(x)
% this is not part of documentation given by help function

    s = sin(x);
    c = cos(x);
    res = abs(s) + abs(c);
end
\end{code}

When you ask for \textbf{\lstinline{help}}, MATLAB prints the comment you provide.

\index{help@\lstinline{help}}

\begin{code}
>> ***help myfunc***
  res = myfunc(x)
  Compute the Manhattan distance from the origin to the
  point on the unit circle with angle (x) in radians.
\end{code}

There are lots of conventions about what should be included
in these comments.  Among other things, it's a good idea to
include:

\begin{description}

\item [Signature] The signature of the function, which includes the name
of the function, the input variable(s), and the output variable(s).

\item [Description] A clear, concise, abstract description of what the function does.
An {\em abstract} description is one that leaves out the
details of {\em how} the function works, and includes only information
that someone using the function needs to know.  You can put additional
comments inside the function that explain the details.

\item [Variables] An explanation of what the input variables mean; for example,
in this case it is important to note that \lstinline{x} is considered
to be an angle in radians.

\item [Conditions] Any preconditions and postconditions.

\end{description}

\index{precondition}
\index{postcondition}

\section{Naming Functions}
%\index{Functions!naming}

There are a few ``gotchas'' that come up when you start defining functions.
The first is that the ``real'' name of your function is determined by the file name, {\em not} by the name you put in the function signature.  As a matter of style, you
should make sure that they are always the same, but if you
make a mistake, or if you change the name of a function, it's
easy to get confused.

\index{function name}
\index{name!function}

In the spirit of making errors on purpose, change the name of
the function \lstinline{myfunc} to \lstinline{something_else}, and
then run it again.

If this is what you put in \emph{myfunc.m}:

\begin{code}
function res = something_else (x)
    s = sin(x);
    c = cos(x);
    res = abs(s) + abs(c);
end
\end{code}
here's what you'll get:

\begin{code}
>> ***y = myfunc(1)***
y = 1.3818

>> ***y = something_else(1)***
Undefined function or variable 'something_else'.
\end{code}

Entering \textbf{\lstinline{myfunc}} still works because that's the name of the file;
\lstinline{something_else} doesn't work because the name of the function is ignored.

The second gotcha is that the name of the file can't have spaces.
For example, if you write a function and rename the file to 
{\em my func.m},
and then try to run it, you get:

\begin{code}
>> ***y = my func(1)***
 y = my func(1)
        |
Error: Unexpected MATLAB expression.
\end{code}
This fails because MATLAB thinks \lstinline{my} and \lstinline{func} are two different
variable names.

The third gotcha is that your function names can collide with built-in
MATLAB functions.  For example, if you create an M-file named {\em sum.m}, and then call \lstinline{sum}, MATLAB might call {\em your} new
function, not the built-in version!  Which one actually gets called
depends on the order of the directories in the search path, and
(in some cases) on the arguments.  As an example, put the following
code in a file named {\em sum.m}:

\index{name collision}
\index{collision!name}

\begin{code}
function res = sum(x)
   res = 7;
end
\end{code}

And then try this:

\begin{code}
>> ***sum(1:3)***

ans = 6

>> ***sum***

ans = 7
\end{code}

In the first case MATLAB used the built-in function; in the second
case it ran your function!  This kind of interaction can be very
confusing.  Before you create a new function, check to see if there is
already a MATLAB function with the same name.  If there is, choose
another name!

\section{Multiple Input Variables}
\label{hypotenuse}

\index{input variable}
\index{variable!input}

Functions can take more than one input variable.
For example, the following function takes two input variables,
\lstinline{a} and \lstinline{b}:

\begin{lstlisting}[caption={A function that computes the sum of squares of two numbers}, label={lst:hyp_function}]
function res = sum_squares(a, b)
    res = a^2 + b^2;
end
\end{lstlisting}
  
This function computes the sum of squares of two numbers, \lstinline{a}
and \lstinline{b}.

If we call it from the Command Window with arguments 3 and 4, we can
confirm that the sum of their squares is 25.

\begin{code}
>> ***ss = sum_squares(3, 4)***
ss = 25
\end{code}

The arguments you provide are assigned to the input variables in
order, so in this case 3 is assigned to \lstinline{a} and 4 is assigned to
\lstinline{b}.  MATLAB checks that you provide the right number of arguments;
if you provide too few, you get

\begin{code}
>> ***ss = sum_squares(3)***
Not enough input arguments.

Error in sum_squares (line 4)
    res = a^2 + b^2;
\end{code}

This error message might be confusing, because it suggests that
the problem is in \lstinline{sum_squares} rather than in the function call.
Keep that in mind when you're debugging.

If you provide too many arguments, you get

\begin{code}
ss = sum_squares(3, 4, 5)
Error using sum_squares
Too many input arguments.
\end{code}

That's a better error message, because it's clear that the problem isn't in the function, it's in the way we're using the function.

\section{Chapter Review}

Now that we know about functions, and all the ways they can go wrong, let's put them to good use.  In the next chapter we'll develop a program that uses several functions to search for Pythagorean triples (and I'll explain what those are).

Here are a few terms in this chapter you might want to remember.

A {\em function} is a named sequence of statements stored in an M-file.
A function can have one or more {\em input variables}, which get their values when the function is called, and {\em output variables}, which return a value from the function to the caller.

The first line of a function definition is its {\em signature}, which
specifies the name of the function, the input variables, and the
output variables.

A {\em silent function} doesn't display anything, generate a figure, or have any other effect other than returning output values.


\section{Exercises}

Before you go on, you might want to work on the following exercise.
\subsection{Exercise~1}
\label{hypotenuse_exercise}
Write a function called \lstinline{hypotenuse} that takes two parameters, \lstinline{a} and \lstinline{b}, that represent the lengths of two sides of a right triangle.  It should assign to \lstinline{res} the length of the third side of the triangle, given by the formula

\[ c = \sqrt{a^2 + b^2}. \]

