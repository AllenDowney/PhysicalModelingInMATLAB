\documentclass[main.tex]{subfiles}

\begin{document}

\chapter{Functions}
\label{chpt:functions}

This chapter introduces the most important idea in computer programming: functions!

\index{function}

A {\bf function} is like a script, except

\begin{itemize}

\item Each function has its own workspace, so any variables defined
inside a function only exist while the function is running, and don't
interfere with variables in other workspaces, even if they have the
same name.

\item Function inputs and outputs are defined carefully to avoid
unexpected interactions.

\end{itemize}

To define a new function, you create an M-file with the name you
want, and put a function definition in it.  For example, to create
a function named {\tt myfunc}, create an M-file named {\tt myfunc.m}
and put the following definition into it.

\index{M-file}
\index{script}
\index{function definition}

\begin{code}
function res = myfunc(x)
    s = sin(x)
    c = cos(x)
    res = abs(s) + abs(c)
end
\end{code}

The first non-comment word of the file has to be {\tt function}, because
that's how MATLAB tells the difference between a script and a function
file.

\index{compound statement}
\index{definition!function}

A function definition is a compound statement.  The first line
is called the {\bf signature} of the function; it defines
the inputs and outputs of the function.  In this case the {\bf input variable} is named {\tt x}.  When this function is called, the
argument provided by the user will be assigned to {\tt x}.

\index{input variable}
\index{variable!input}

\index{output variable}
\index{variable!output}

The {\bf output variable} is named {\tt res}, which is short for
``result''.  You can call the output variable whatever you want, but
as a convention, I like to call it {\tt res}.  Usually the last
thing a function does is assign a value to the output variable.

\index{res@{\tt res}}
\index{ans@{\tt ans}}

Once you have defined a new function, you call it the same way you
call built-in MATLAB functions.  If you call the function as a statement,
MATLAB puts the result into {\tt ans}:

\begin{code}
>> myfunc(1)

s = 0.84147098480790

c = 0.54030230586814

res = 1.38177329067604

ans = 1.38177329067604
\end{code}

But it is more common (and better style) to assign the result to
a variable:

\begin{code}
>> y = myfunc(1)

s = 0.84147098480790

c = 0.54030230586814

res = 1.38177329067604

y = 1.38177329067604
\end{code}

While you are debugging a new function, you might want to display
intermediate results like this, but once it is working, you will want
to add semi-colons to make it a {\bf silent function}.  Most built-in
functions are silent; they compute a result, but they don't display
anything (except sometimes warning messages).

\index{silent function}
\index{function!silent}
\index{workspace}

Each function has its own workspace, which is created when the
function starts and destroyed when the function ends.  If you try to
access (read or write) the variables defined inside a function, you
will find that they don't exist.

\begin{code}
>> clear
>> y = myfunc(1);
>> who
Your variables are: y

>> s
Undefined function or variable 's'.
\end{code}

The only value from the function that you can access is the result,
which in this case is assigned to {\tt y}.

If you have variables named {\tt s} or {\tt c} in your workspace
before you call {\tt myfunc}, they will still be there when the
function completes.

\begin{code}
>> s = 1;
>> c = 1;
>> y = myfunc(1);
>> s, c

s = 1
c = 1
\end{code}

So inside a function you can use whatever variable names you
want without worrying about collisions.

\index{name collision}
\index{collision!name}


\section{Documentation}

\index{documentation}
\index{comment}

At the beginning of every function file, you should include a comment
that explains what the function does:

\index{Documentation!functions}.

\begin{code}
% res = silent_myfunc(x)
% Compute the Manhattan distance from the origin to the
% point on the unit circle with angle (x) in radians.

function res = silent_myfunc(x)
% this is not part of documentation given by help function

    s = sin(x);
    c = cos(x);
    res = abs(s) + abs(c);
end
\end{code}

When you ask for {\tt help}, MATLAB prints the comment you provided at the top of the file.

\index{help@{\tt help}}

\begin{code}
>> help silent_myfunc
  res = silent_myfunc(x)
  Compute the Manhattan distance from the origin to the
  point on the unit circle with angle (x) in radians.
\end{code}

There are lots of conventions about what should be included
in these comments.  Among other things, it is a good idea to
include

\begin{itemize}

\item The signature of the function, which includes the name
of the function, the input variable(s) and the output variable(s).

\item A clear, concise, abstract description of what the function does.
An {\bf abstract} description is one that leaves out the
details of {\em how} the function works, and includes only information
that someone using the function needs to know.  You can put additional
comments inside the function that explain the details.

\item An explanation of what the input variables mean; for example,
in this case it is important to note that {\tt x} is considered
to be an angle in radians.

\item Any preconditions and postconditions.

\end{itemize}

\index{precondition}
\index{postcondition}


\section{What could go wrong?}
\index{Functions!naming}

There are a few ``gotchas'' that come up when you start defining functions.
The first is that the ``real'' name of your function is determined by the file name, {\em not} by the name you put in the function signature.  As a matter of style, you
should make sure that they are always the same, but if you
make a mistake, or if you change the name of a function, it is
easy to get confused.

\index{function name}
\index{name!function}

In the spirit of making errors on purpose, change the name of
the function in \verb"myfunc" to \verb"something_else", and
then run it again.

If this is what you put in \verb"myfunc.m":

\begin{code}
function res = something_else (x)
    s = sin(x);
    c = cos(x);
    res = abs(s) + abs(c);
end
\end{code}

Here's what you'll get:

\begin{code}
>> y = myfunc(1)
y = 1.3818

>> y = something_else(1)
Undefined function or variable 'something_else'.
\end{code}

{\tt myfunc} still works because that's the name of the file.
\verb"something_else" doesn't work because the name of the function is ignored.

The second gotcha is that the name of the file can't have spaces.
For example, if you write a function and rename the file to 
{\tt my func.m},
and then try to run it, you get:

\begin{code}
>> y = my func(1)
 y = my func(1)
        |
Error: Unexpected MATLAB expression.
\end{code}

The third gotcha is that your function names can collide with built-in
MATLAB functions.  For example, if you create an M-file named {\tt sum.m}, and then call {\tt sum}, MATLAB might call {\em your} new
function, not the built-in version!  Which one actually gets called
depends on the order of the directories in the search path, and
(in some cases) on the arguments.  As an example, put the following
code in a file named {\tt sum.m}:

\index{name collision}
\index{collision!name}

\begin{code}
function res = sum(x)
   res = 7;
end
\end{code}

And then try this:

\begin{code}
>> sum(1:3)

ans = 6

>> sum

ans = 7
\end{code}

In the first case MATLAB used the built-in function; in the second
case it ran your function!  This kind of interaction can be very
confusing.  Before you create a new function, check to see if there is
already a MATLAB function with the same name.  If there is, choose
another name!


\section{Multiple input variables}
\label{sect:hypotenuse}

\index{input variable}
\index{variable!input}

Functions can, and often do, take more than one input variable.
For example, the following function takes two input variables,
{\tt a} and {\tt b}:

\begin{code}
function res = hypotenuse(a, b)
    res = (a^2 + b^2) ^ (1/2);
end
\end{code}
  
This function computes the length of the hypotenuse of a right
triangle if the lengths of the adjacent sides are {\tt a}
and {\tt b}.

If we call it from the {\sf Command Window} with arguments 3 and 4, we can
confirm that the length of the third side is 5.

\begin{code}
>> c = hypotenuse(3, 4)
c = 5
\end{code}

The arguments you provide are assigned to the input variables in
order, so in this case 3 is assigned to {\tt a} and 4 is assigned to
{\tt b}.  MATLAB checks that you provide the right number of arguments;
if you provide too few, you get

\begin{code}
>> c = hypotenuse(3)
Not enough input arguments.

Error in hypotenuse (line 2)
    res = (a^2 + b^2) ^ (1/2);
\end{code}

This error message is slightly confusing, because it suggests that
the problem is in {\tt hypotenuse} rather than in the function call.
Keep that in mind when you are debugging.

If you provide too many arguments, you get

\begin{code}
>> c = hypotenuse(3, 4, 5)
Error using hypotenuse
Too many input arguments.
\end{code}

Which is a better message.


\section{Logical functions}

In Section~\ref{sect:logop} we used logical operators to compare values.
MATLAB also provides {\bf logical functions} that check for certain
conditions and return logical values: 1 for ``true'' and 0 for ``false''.

\index{logical function}
\index{function!logical}
\index{logical value}
\index{value!logical}

For example, {\tt isprime} checks to see whether a number is prime.

\begin{code}
>> isprime(17)
ans = 1

>> isprime(21)
ans = 0
\end{code}

The functions {\tt isscalar} and {\tt isvector} check whether
a value is a scalar or vector.

To check whether a value you have computed is an integer, you might
be tempted to use {\tt isinteger}.  But that would be wrong, so very
wrong.  {\tt isinteger} checks whether a value belongs to one of
the integer types (a topic we have not discussed); it doesn't check
whether a floating-point value happens to be integral.

\begin{code}
>> c = hypotenuse(3, 4)
c = 5

>> isinteger(c)
ans = 0
\end{code}

To do that, we have to write our own logical function, which
we'll call {\tt isintegral}:

\begin{code}
function res = isintegral(x)
    if round(x) == x
        res = 1;       % or... res = true
    else
        res = 0;       % or... res = false
    end
end
\end{code}

This function is good enough for many applications, but remember
that floating-point values are only approximately right:
sometimes the approximation is an integer when the actual value is not; 
sometimes the approximation is not an integer when the actual value is.

\index{floating-point}


\section{Incremental development}
\label{sect:increxample}

Suppose we want to write a program to search for ``Pythagorean
triples'': sets of integral values, like 3, 4, and 5,
that are the lengths of the sides of a right triangle.  In other
words, we would like to find integral values $a$, $b$, and $c$ such
that $a^2 + b^2 = c^2$.

\index{Pythagorean triple}
\index{incremental development}

Here are the steps we will follow to develop the program incrementally:

\begin{itemize}

\item Write a script named {\tt find\_triples} and start with a simple
statement like {\tt x=5}.

\item Write a loop that enumerates values of $a$ from 1 to 3, and
displays them.

\item Write a nested loop that enumerates values of $b$ from 1 to 4,
and displays them.

\item Inside the loop, call {\tt hypotenuse} to compute $c$ and
display it.

\item Use {\tt isintegral} to check whether $c$ is an integral
value.

\item Use an if statement to print only the triples $a$, $b$, and $c$
that pass the test.

\item Transform the script into a function.

\item Generalize the function to take input variables that
specify the range to search.

\end{itemize}

Starting with {\tt x=5} might seem
silly, but if you start simple and add a little bit at a time, you
will avoid a lot of debugging.

Here's the second draft:

\begin{code}
for a=1:3
    a
end
\end{code}

At each step, the program is testable: it produces output (or another
visible effect) that you can check.


\section{Nested loops}

\index{nested loop}
\index{loop!nested}

The third draft contains a nested loop:

\begin{code}
for a=1:3
    a
    for b=1:4
        b
    end
end
\end{code}

The inner loop gets executed 3 times, once for each value of {\tt a},
so here's what the output looks like (I adjusted the spacing to make
the structure clear):

\begin{code}
>> find_triples

a = 1   b = 1
        b = 2
        b = 3
        b = 4

a = 2   b = 1
        b = 2
        b = 3
        b = 4

a = 3   b = 1
        b = 2
        b = 3
        b = 4
\end{code}

The next step is to compute $c$ for each pair of values $a$ and $b$.

\begin{code}
for a=1:3
    for b=1:4
        c = hypotenuse(a, b);
        [a, b, c]
    end
end
\end{code}

To display the values of {\tt a}, {\tt b}, and {\tt c}, I store them in a vector; here's what the output looks like:

\begin{code}
>> find_triples

ans = 1.0000    1.0000    1.4142
ans = 1.0000    2.0000    2.2361
ans = 1.0000    3.0000    3.1623
ans = 1.0000    4.0000    4.1231
ans = 2.0000    1.0000    2.2361
ans = 2.0000    2.0000    2.8284
ans = 2.0000    3.0000    3.6056
ans = 2.0000    4.0000    4.4721
ans = 3.0000    1.0000    3.1623
ans = 3.0000    2.0000    3.6056
ans = 3.0000    3.0000    4.2426
ans = 3         4         5
\end{code}

You might notice that we are wasting some effort here.
After checking $a=1$ and $b=2$, there is no point in checking
$a=2$ and $b=1$.  We can eliminate the extra work by adjusting the
range of the second loop:

\begin{code}
for a=1:3
    for b=a:4
        c = hypotenuse(a, b);
        [a, b, c]
    end
end
\end{code}

If you are following along, run this version to make sure it has
the expected effect.


\section{Conditions and flags}

\index{condition}
\index{flag}

The next step is to check for integral values of $c$.  This
loop calls {\tt isintegral} and prints the resulting logical
value.

\begin{code}
for a=1:3
    for b=a:4
        c = hypotenuse(a, b);
        flag = isintegral(c);
        [c, flag]
    end
end
\end{code}

By not displaying {\tt a} and {\tt b} I made it easy to scan the
output to make sure that the values of {\tt c} and {\tt flag}
look right.

\begin{code}
>> find_triples

ans = 1.4142         0
ans = 2.2361         0
ans = 3.1623         0
ans = 4.1231         0
ans = 2.8284         0
ans = 3.6056         0
ans = 4.4721         0
ans = 4.2426         0
ans = 5              1
\end{code}

The next step is to use {\tt flag} to display only the successful
triples:

\begin{code}
for a=1:3
    for b=a:4
        c = hypotenuse(a, b);
        flag = isintegral(c);
        if flag
            [a, b, c]
        end
    end
end
\end{code}

Now the output is minimal:

\begin{code}
>> find_triples

ans = 3     4     5
\end{code}




\section{Encapsulation and generalization}

As a script, this program has the side-effect of assigning values to
{\tt a}, {\tt b}, {\tt c}, and {\tt flag}, which would make it hard to
use if any of those names were in use.  
By wrapping the code in a function, we can avoid name collisions; this process is called {\bf encapsulation} because it isolates this program from the workspace.

\index{encapsulation}
\index{generalization}

The first draft of the function takes no input variables:

\begin{code}
function res = find_triples ()
    for a=1:3
        for b=a:4
            c = hypotenuse(a, b);
            flag = isintegral(c);
            if flag
                [a, b, c]
            end
        end
    end
end
\end{code}

The empty parentheses in the signature are not strictly necessary, but
they make it apparent that there are no input variables.  Similarly,
when I call the new function, I like to use parentheses to remind me
that it is a function, not a script:

\begin{code}
>> find_triples()
\end{code}

The output variable isn't strictly necessary, either; it
never gets assigned a value.  But I put it there as a matter of
habit, and also so my function signatures all have the same structure.

\index{output variable}
\index{variable!output}

The next step is to generalize this function by adding input
variables.  The natural generalization is to replace the constant
values 3 and 4 with a variable so we can search an arbitrarily large
range of values.

\begin{code}
function res = find_triples (n)
    for a=1:n
        for b=a:n
            c = hypotenuse(a, b);
            flag = isintegral(c);
            if flag
                [a, b, c]
            end
        end
    end
end
\end{code}

Here are the results for the range from 1 to 15:

\begin{code}
>> find_triples(15)

ans = 3     4     5
ans = 5    12    13
ans = 6     8    10
ans = 8    15    17
ans = 9    12    15
\end{code}

Some of these are more interesting than others.  The triples
$5,12,13$ and $8,15,17$ are ``new,'' but the others are just
multiples of the $3,4,5$ triangle we already knew.


%\section{A misstep}
%
%When you change the signature of a function, you have to change
%the places that call the function, too.  For example, suppose
%you decide to add a third input variable to {\tt hypotenuse}:
%
%% wcs -- talk about copying from a different file using copyfile
%% (and waiting after copying the file)
%% ABD: not sure what you are getting at here.
%
%\begin{code}
%function res = hypotenuse(a, b, d)
%    res = (a^d + b^d) ^ (1/d);
%end
%\end{code}
%
%When {\tt d} is 2, the function does the same thing it did before.  There is no practical reason to generalize the function in this way; it's just
%an example.  Now when you run {\tt find\_triples}, you get:
%
%\begin{code}
%>> find_triples(20)
%Not enough input arguments.
%
%Error in hypotenuse (line 2)
%    res = (a^d + b^d)^(1/d);
%
%Error in find_triples (line 7)
%            c = hypotenuse(a, b);
%\end{code}
%
%So that makes it pretty easy to find the error.  
%
%% ABD: I removed the 8th theorem of debugging here because the error
%% message no longer makes a suggestion, right or wrong


\section{{\tt continue}}

\index{continue@{\tt continue}}

As a final improvement, let's modify the function so it only
displays the ``lowest'' of each Pythagorean triple, and not the
multiples.

The simplest way to eliminate the multiples is to check whether
$a$ and $b$ share a common factor.  If they do, dividing both
by the common factor yields a smaller, similar triangle that has
already been checked.

\index{gcd function@{\tt gcd} function}
\index{function!{\tt gcd}}

MATLAB provides a {\tt gcd} function that computes the greatest common
divisor of two numbers.  If the result is greater than 1,
$a$ and $b$ share a common factor and we can use the {\tt continue}
statement to skip to the next pair:

\begin{code}
function res = find_triples (n)
    for a=1:n-1
        for b=a:n
            if gcd(a,b) > 1
                continue
            end
            c = hypotenuse(a, b);
            if isintegral(c)
                [a, b, c]
            end
        end
    end
end
\end{code}

{\tt continue} causes the program to end the current iteration
immediately, jump to the top of the loop, and ``continue'' with the next iteration.

In this case, since there are two loops, it might not be obvious which loop to jump to, but the rule is to jump to the inner-most loop (which is what we want).

\index{flag}

I also simplified the program slightly by eliminating {\tt flag} and using {\tt isintegral} as the condition of the {\tt if} statement.

Here are the results with {\tt n=40}:

\begin{code}
>> find_triples(40)

ans =  3     4     5
ans =  5    12    13
ans =  7    24    25
ans =  8    15    17
ans =  9    40    41
ans = 12    35    37
ans = 20    21    29
\end{code}





\section{Mechanism and leap of faith}

\index{function call}
\index{leap of faith}

Let's review the sequence of steps that occur when you call a function:

\begin{enumerate}

\item Before the function starts running, MATLAB creates a new
workspace for it.
\index{workspace}

\item MATLAB evaluates each of the arguments and assigns
the resulting values, in order, to the input variables (which
live in the {\em new} workspace).

\item The body of the code executes.  Somewhere in the body
(often the last line) a value gets assigned to the output variable.

\item The function's workspace is destroyed; the only thing
that remains is the value of the output variable and any side
effects the function had (like displaying values or creating
a figure).

\item The program resumes from where it left off.  The value
of the function call is the value of the output variable.

\end{enumerate}

When you are reading a program and you come to a function call,
there are two ways to interpret it:

\begin{itemize}

\item You can think about the mechanism I just described,
and follow the execution of the program into the function and back, or

\item You can take the ``leap of faith'': assume that the function
works correctly, and go on to the next statement after the function call.

\end{itemize}

When you use built-in functions, it is natural to take the leap
of faith, in part because you expect that most
MATLAB functions work, and in part because you don't
generally have access to the code in the body of the function.

But when you start writing your own functions, you will probably
find yourself following the ``flow of execution''.  This can
be useful while you are learning, but as you gain experience, you
should get more comfortable with the idea of writing a function,
testing it to make sure it works, and then forgetting about the
details of how it works.

\index{flow of execution}
\index{abstraction}

Forgetting about details is called {\bf abstraction}; in the context
of functions, abstraction means forgetting about {\em how} a function
works, and just assuming (after appropriate testing) that it works.


\section{Why functions?}

\index{function!reasons for}

For many people, it takes some time to get comfortable with functions.  If you are one of them, you might be tempted to avoid functions, and sometimes you can get by without them.

But experienced programmers use functions extensively, for several good reasons:

\begin{itemize}

\item Each function has its own workspace, so using functions helps
avoid name collisions.

\item Functions lend themselves to incremental development: you can
debug the body of the function first (as a script), then encapsulate
it as a function, and then generalize it by adding input variables.

\item Functions allow you to divide a large problem into small
pieces, work on the pieces one at a time, and then assemble a
complete solution.

\item Once you have a function working, you can forget about the
details of how it works and concentrate on what it does.  This
process of abstraction is an important tool for managing the
complexity of large programs.

\end{itemize}

Another reason to use functions is that many of the tools provided by MATLAB require them.
For example, in the next chapter we will use {\tt fzero} to find solutions of nonlinear equations.  Later we will use {\tt ode45} to approximate solutions to
differential equations.


\section{Glossary}

\begin{description}

\item[side-effect:] An effect, like modifying the workspace, that
is not the primary purpose of a script.

\item[input variable:] A variable in a function that gets its value,
when the function is called, from one of the arguments.

\item[output variable:] A variable in a function that is used to
return a value from the function to the caller.

\item[signature:] The first line of a function definition, which
specifies the names of the function, the input variables and the
output variables.

\item[silent function:] A function that doesn't display anything
or generate a figure, or have any other side-effects.

\item[logical function:] A function that returns a logical value
(1 for ``true'' or 0 for ``false'').

\item[encapsulation:] The process of wrapping part of a program in
a function in order to limit interactions (including name collisions)
between the function and the rest of the program.

\item[generalization:] Making a function more versatile by replacing
specific values with input variables.

\item[abstraction:] The process of ignoring the details of how
a function works in order to focus on a simpler model of what the
function does.

\end{description}


\section{Exercises}

\begin{ex}
\index{Fibonacci number}
\index{Pythagorean triple}

There is an interesting connection between Fibonacci numbers and
Pythagorean triples.  If $F$ is a Fibonacci sequence,

\begin{equation}
(F_i F_{i+3}, 2 F_{i+1} F_{i+2}, F_{i+1}^2 + F_{i+2}^2 )
\end{equation}

is a Pythagorean triple, for all $i \ge 1$.

Write a function named {\tt fib\_triple} that
takes {\tt n} as an input variable and computes 
the first {\tt n} Fibonacci numbers, and stores them in a vector,
then checks whether this formula produces a Pythagorean triple for all {\tt i} such that $ 1 \leq i \leq n-3 $

% fib_triple.m
\end{ex}



% chap06


\end{document}
