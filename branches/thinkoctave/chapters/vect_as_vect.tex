

% chap12 - Vectors as vectors
% Last edited:

\chapter{Vectors as vectors}

\section{What's a vector?}

The word ``vector'' means different things to different people.
In Octave, a vector is a matrix that has either one row or one
column. So far we have used Octave vectors to represent

\begin{description}

\item[sequences:] A sequence is a set of values identified by
integer indices; it is natural to store the elements of the
sequence as elements of a Octave vector.

\item[state vectors:] A state vector is a set of values that
describes the state of a physical system. When you call
{\tt ode45}, you give it in the initial conditions in a state
vector. Then when {\tt ode45} calls your rate function, it
gives you a state vector.

\item[discrete maps:] If you have two vectors with the same
length, you can think of them as a mapping from the elements
of one vector to the corresponding elements of the other. For
example, in Section~\ref{rats}, the results from {\tt ode45}
are vectors, {\tt T} and {\tt Y}, that represent a mapping
from the time values in {\tt T} to the population values in {\tt Y}.

\end{description}

In this chapter we will see another use of Octave vectors:
representing spatial vectors. A spatial vector is a value that
represents a multidimensional physical quantity like position,
velocity, acceleration or force\footnote{See
\url{http://en.wikipedia.org/wiki/Vector_(spatial)}.}.

These quantities cannot be described with a
single number because they contain multiple components. For example,
in a 3-dimensional Cartesian coordinate space, it takes three numbers
to specify a position in space; they are usually called $x$, $y$ and
$z$ coordinates. As another example, in 2-dimensional polar
coordinates, you can specify a velocity with two numbers, a
magnitude and an angle, often called $r$ and $\theta$.

It is convenient to represent spatial vectors using Octave vectors
because Octave knows how to perform most of the vector
operations you need for physical modeling. For example,
suppose that you are given the velocity of a baseball in
the form of a Octave vector with two elements, $v_x$ and $v_y$,
which are the components of velocity in the $x$ and $y$ directions.

\begin{verbatim}
octave:1> V = [30, 40]    % velocity in m/s
\end{verbatim}

And suppose you are asked to compute the total acceleration of
the ball due to drag and gravity. In math notation, the force
due to drag is

\[ F_d = -\frac{1}{2} ~ \rho ~ v^2 ~ A ~ C_d ~ \hat{V}  \]

where $V$ is a spatial vector representing velocity, $v$ is the magnitude
of the velocity (sometimes called ``speed''), and $\hat{V}$ is a unit
vector in the direction of the velocity vector. The other terms,
$\rho$, $A$ and $C_d$, are scalars.

The magnitude of a vector is the square root of the sum of the squares
of the elements. You could compute it with {\tt hypotenuse} from
Section~\ref{hypotenuse}, or you could use the Octave function {\tt
norm} ({\bf norm} is another name\footnote{Magnitude is also called
``length'' but I will avoid that term because it gets confused with
the {\tt length} function, which returns the number of elements in a
Octave vector.} for the magnitude of a vector):

\begin{verbatim}
octave:1> v = norm(V)

v = 50
\end{verbatim}

$\hat{V}$ is a {\bf unit vector}, which means it should have norm 1,
and it should point in the same direction as $V$. The simplest
way to compute it is to divide $V$ by its own norm.

\begin{verbatim}
octave:1> Vhat = V / v

Vhat = 0.6 0.8
\end{verbatim}

Then we can confirm that the norm of $\hat{V}$ is 1:

\begin{verbatim}
octave:1> norm(Vhat)

ans = 1
\end{verbatim}

To compute $F_d$ we just multiply the scalar terms by $\hat{V}$.

\begin{verbatim}
Fd = - 1/2 * C * rho * A * v^2 * Vhat
\end{verbatim}

Similarly, we can compute acceleration by dividing the vector
$F_d$ by the scalar $m$.

\begin{verbatim}
Ad = Fd / m
\end{verbatim}

To represent the acceleration of gravity, we create a vector 
with two components:

\begin{verbatim}
Ag = [0; -9.8]
\end{verbatim}

The $x$ component of gravity is 0; the $y$ component is $-9.8 m/s^2$.

Finally we compute total acceleration by adding vector
quantities:

\begin{verbatim}
A = Ag + Ad;
\end{verbatim}

One nice thing about this computation is that we didn't have to
think much about the components of the vectors. By treating
spatial vectors as basic quantities, we can express complex computations
concisely.


\section{Dot and cross products}

Multiplying a vector by a scalar is a straightforward operation;
so is adding two vectors. But multiplying two vectors is more
subtle. It turns out that there are two vector operations that
resemble multiplication: {\bf dot product}
and {\bf cross product}.

The dot product of vectors $A$ and $B$ is a scalar:

\[ d = a b \cos \theta \]
%
where $a$ is the magnitude of $A$, $b$ is the magnitude of $B$,
and $\theta$ is the angle between the vectors. We already know
how to compute magnitudes, and you could probably figure out
how to compute $\theta$, but you don't have to. Octave provides
a function, {\tt dot}, that computes dot products.

\begin{verbatim}
d = dot(A, B)
\end{verbatim}

{\tt dot} works in any number of dimensions, as long as {\tt A}
and {\tt B} have the same number of elements.

If one of the operands is a unit vector, you can use the dot
product to compute the component of a vector $A$ that is in
the direction of a unit vector, $\hat{i}$:

\begin{verbatim}
s = dot(A, ihat)
\end{verbatim}

In this example, $s$ is the {\bf scalar projection} of $A$
onto $\hat{i}$. The {\bf vector projection} is the vector
that has magnitude $s$ in the direction of $\hat{i}$:

\begin{verbatim}
V = dot(A, ihat) * ihat
\end{verbatim}

The cross product of vectors $A$ and $B$ is a vector whose direction
is perpendicular to $A$ and $B$ and whose magnitude is

\[ c = a b \sin \theta \]
%
where (again) $a$ is the magnitude of $A$, $b$ is the magnitude of
$B$, and $\theta$ is the angle between the vectors. Octave provides
a function, {\tt cross}, that computes cross products.

\begin{verbatim}
C = cross(A, B)
\end{verbatim}

{\tt cross} only works for 3-dimensional vectors; the result is
a 3-dimensional vector.

A common use of {\tt cross} is to compute torques. If you represent
a moment arm $R$ and a force $F$ as 3-dimensional vectors, then
the torque is just

\begin{verbatim}
Tau = cross(R, F)
\end{verbatim}

If the components of {\tt R} are in meters and the components
of {\tt F} are in Newtons, then the torques in {\tt Tau} are
in Newton-meters.



\section{Celestial mechanics}

Modeling celestial mechanics is a good opportunity
to compute with spatial vectors.
Imagine a star with mass $m_1$ at a point in space described by the
vector $P_1$, and a planet with mass $m_2$ at point $P_2$. The
magnitude of the gravitational force\footnote{See
\url{http://en.wikipedia.org/wiki/Gravity}} between them is

\[ f_g = G \frac{m_1 m_2}{r^2} \]
%
where $r$ is the distance between them and $G$ is the universal
gravitational constant, which is about $6.67 \times 10^{-11} N m^2 /
kg^2$. Remember that this is the appropriate value of $G$ only if the
masses are in kilograms, distances in meters, and forces in Newtons.

The direction of the force on the star at $P_1$ is in the
direction toward $P_2$. We can compute relative direction by
subtracting vectors; if we compute {\tt R = P2 - P1}, then
the direction of {\tt R} is from {\tt P1} to {\tt P2}.

The distance between the planet and star is the length of $R$:

\begin{verbatim}
r = norm(R)
\end{verbatim}

The direction of the force on the star is $\hat{R}$:

\begin{verbatim}
rhat = R / r
\end{verbatim}

\begin{ex}
Write a sequence of Octave statements that computes {\tt F12}, a vector
that represents the force on the star due to the planet, and {\tt
F21}, the force on the planet due to the star.
\end{ex}

\begin{ex}
Encapsulate these statements in a function named {\tt
gravity\_force\_func} that takes {\tt P1}, {\tt m1}, {\tt P2}, and
{\tt m2} as input variables and returns {\tt F12}.
\end{ex}

\begin{ex}
\label{jupiter}
Write a simulation of the orbit of Jupiter around the Sun. The mass
of the Sun is about $2.0 \times 10^{30}$ kg. You can get the mass of
Jupiter, its distance from the Sun and orbital velocity from
\url{http://en.wikipedia.org/wiki/Jupiter}. Confirm that it takes
about 4332 days for Jupiter to orbit the Sun.
\end{ex}

\section{Animation}

Animation is a useful tool for checking the results of a physical
model. If something is wrong, animation can make it obvious.
There are two ways to do animation in Octave. One is to use
{\tt getframe} to capture a series of images and {\tt movie} to
play them back.
The more informal way is to draw a series of plots.
Here is an example I wrote for Exercise~\ref{jupiter}:

\begin{verbatim}
function animate_func(T,M)
  % animate the positions of the planets, assuming that the
  % columns of M are x1, y1, x2, y2.
  X1 = M(:,1);
  Y1 = M(:,2);
  X2 = M(:,3);
  Y2 = M(:,4);

  minmax = [min([X1;X2]), max([X1;X2]), min([Y1;Y2]), max([Y1;Y2])];

  for i=1:length(T)
    clf;
    axis(minmax);
    hold on;
    draw_func(X1(i), Y1(i), X2(i), Y2(i));
    drawnow;
  end
end
\end{verbatim}

The input variables are the output from {\tt ode45}, a vector
{\tt T} and a matrix {\tt M}. The columns of {\tt M} are the
positions and velocities of the Sun and Jupiter, so
{\tt X1} and {\tt Y1} get the coordinates of the Sun;
{\tt X2} and {\tt Y2} get the coordinates of Jupiter.

{\tt minmax} is a vector of four elements which is used inside
the loop to set the axes of the figure. This is necessary because
otherwise Octave scales the figure each time through the loop,
so the axes keep changing, which makes the animation hard
to watch.

Each time through the loop, {\tt animate\_func} uses {\tt clf}
to clear the figure and {\tt axis} to reset the axes. {\tt hold
on} makes it possible to put more than one plot onto the same
axes (otherwise Octave clears the figure each time you call
{\tt plot}).

Each time through the loop, we have to call {\tt drawnow} so
that Octave actually displays each plot. Otherwise it waits
until you finish drawing all the figures and {\em then} updates
the display.

{\tt draw\_func} is the function that actually makes the
plot:

\begin{verbatim}
function draw_func(x1, y1, x2, y2)
  plot(x1, y1, 'r.', 'MarkerSize', 50);
  plot(x2, y2, 'b.', 'MarkerSize', 20);
end
\end{verbatim}

The input variables are the position of the Sun and Jupiter.
{\tt draw\_func} uses {\tt plot} to draw
the Sun as a large red marker and Jupiter as a smaller blue one.

\begin{ex}
To make sure you understand how {\tt animate\_func} works,
try commenting out some of the lines to see what happens.
\end{ex}

One limitation of this kind of animation is that the speed
of the animation depends on how fast your computer can generate
the plots. Since the results from {\tt ode45} are usually not
equally spaced in time, your animation might slow down where
{\tt ode45} takes small time steps and speed up where the time
step is larger.

There are two ways to fix this problem:

\begin{enumerate}

\item When you call {\tt ode45} you can give it a vector of
points in time where it should generate estimates. Here is
an example:

\begin{verbatim}
end_time = 1000;
step = end_time/200;
[T, M] = ode45(@rate_func, [0:step:end_time], W);
\end{verbatim}

The second argument is a range vector that goes from 0 to 1000 with a
step size determined by {\tt step}. Since {\tt step} is {\tt
end\_time/200}, there will be about 200 rows in {\tt T} and {\tt M}
(201 to be precise).

This option does not affect the accuracy of the results; {\tt ode45}
still uses variable time steps to generate the estimates, but then it
interpolates them before returning the results.

\item You can use {\tt pause} to play the animation in
real time. After drawing each frame and calling
{\tt drawnow}, you can compute the time
until the next frame and use {\tt pause} to wait:

\begin{verbatim}
dt = T(i+1) - T(i);
pause(dt);
\end{verbatim}

A limitation of this method is that it ignores the time required to
draw the figure, so it tends to run slow, especially if the figure is
complex or the time step is small.

\end{enumerate}

\begin{ex}
Use {\tt animate\_func} and {\tt draw\_func} to vizualize your
simulation of Jupiter. Modify it so it shows one day of simulated
time in 0.001 seconds of real time---one revolution should take
about 4.3 seconds.
\end{ex}


\section{Conservation of Energy}

A useful way to check the accuracy of an ODE solver is to
see whether it conserves energy. For planetary
motion, it turns out that {\tt ode45} does not.

The kinetic energy of a moving body is $m v^2 / 2$; the
kinetic energy of a solar system is the total kinetic
energy of the planets and sun.
The potential energy of a sun with mass $m_1$ and a
planet with mass $m_2$ and a distance $r$ between them is

\[ U = -G \frac{m_1 m_2}{r} \]
%

\begin{ex}
Write a function called {\tt energy\_func} that takes the output of
your Jupiter simulation, {\tt T} and {\tt M}, and computes the total
energy (kinetic and potential) of the system for each estimated
position and velocity. Plot the result as a function of time and
confirm that it decreases over the course of the simulation. Your
function should also compute the relative change in energy, the
difference between the energy at the beginning and end, as a
percentage of the starting energy.
\end{ex}

You can reduce the rate of energy loss by decreasing {\tt ode45}'s
tolerance option using {\tt odeset} (see Section~\ref{events}):

\begin{verbatim}
options = odeset('RelTol', 1e-5);
[T, M] = ode45(@rate_func, [0:step:end_time], W, options);
\end{verbatim}
%
The name of the option is {\tt RelTol} for ``relative tolerance.''
The default value is {\tt 1e-3} or 0.001. Smaller values
make {\tt ode45} less ``tolerant,'' so it does more work to
make the errors smaller. 

\begin{ex}
Run {\tt ode45} with a range of values for {\tt RelTol} and confirm
that as the tolerance gets smaller, the rate of energy loss
decreases.
\end{ex}

\begin{ex}
Run your simulation with one of the other ODE solvers Octave provides
and see if any of them conserve energy.
\end{ex}

% \section{Scaling}

% \section{Polar coordinates}

\section{What is a model for?}

In Section~\ref{modeling} I defined a ``model'' as a simplified
description of a physical system, and said that a good model
lends itself to analysis and simulation, and makes predictions
that are good enough for the intended purpose.

Since then, we have seen a number of examples; now we can
say more about what models are for. The goals of a model tend
to fall into three categories.
 
\begin{description}

\item[prediction:] Some models make predictions about physical
systems. As a simple example, the duck model in
Exercise~\ref{duck} predicts the level a duck floats at. At the other
end of the spectrum, global climate models try to predict the weather
tens or hundreds of years in the future.

\item[design:] Models are useful for engineering design, especially
for testing the feasibility of a design and for optimization. For
example, in Exercise~\ref{golf} you were asked to design the golf
swing with the perfect combination of launch angle, velocity and spin.

\item[explanation:] Models can answer scientific questions. For
example, the Lotka-Volterra model in Section~\ref{lotka} offers a
possible explanation of the dynamics of animal populations systems in
terms of interactions between predator and prey species. 

\end{description}

The exercises at the end of this chapter include one model of
each type.


\section{Glossary}

\begin{description}

\item[spatial vector:] A value that represents a
multidimensional physical quantity like position, velocity,
acceleration or force.

\item[norm:] The magnitude of a vector. Sometimes called ``length,''
but not to be confused with the number of elements in a Octave
vector.

\item[unit vector:] A vector with norm 1, used to indicate
direction.

\item[dot product:] A scalar product of two vectors, proportional
to the norms of the vectors and the cosine of the angle between them.

\item[cross product:] A vector product of two vectors with norm
proportional to the norms of the vectors and the sine of the angle
between them, and direction perpendicular to both.

\item[projection:] The component of one vector that is in the
direction of the other (might be used to mean ``scalar projection'' or
``vector projection'').

\end{description}


\section{Exercises}

\begin{ex}
If you put two identical bowls of water into a freezer, one at
room temperature and one boiling, which one freezes first?

Hint: you might want to do some research on the Mpemba effect.

% You might have to do some research on thermal expansion\footnote{See
% \url{http://en.wikipedia.org/wiki/Thermal_expansion}.},
% cooling\footnote{See
% \url{http://en.wikipedia.org/wiki/Heat_conduction}.},
% evaporation\footnote{See
% \url{http://en.wikipedia.org/wiki/Evaporation}.} and
% freezing\footnote{See \url{http://en.wikipedia.org/wiki/Freezing}.},
% and think about which factors you have to include in the model and
% which you can ignore.
\end{ex}

\begin{ex}
You have been asked to design a new skateboard ramp; unlike a typical
skateboard ramp, this one is free to pivot about a support point.
Skateboarders approach the ramp on a flat surface and then coast up
the ramp; they are not allowed to put their feet down while on the
ramp. If they go fast enough, the ramp will rotate and they will
gracefully ride down the rotating ramp. Technical and artistic
display will be assessed by the usual panel of talented judges.

Your job is to design a ramp that will allow a rider to accomplish
this feat, and to create a physical model of the system, a
simulation that computes the behavior of a rider on the ramp, and an
animation of the result.
\end{ex}

\begin{ex}
\label{binary}

A binary star system contains two stars orbiting each other and
sometimes planets that orbit one or both stars\footnote{See
\url{http://en.wikipedia.org/wiki/Binary_star}.}. In a binary
system, some orbits are ``stable'' in the sense that a planet can stay
in orbit without crashing into one of the stars or flying off into
space.

Simulation is a useful tool for investigating the nature of these
orbits, as in Holman, M.J. and P.A. Wiegert, 1999, ``Long-Term Stability
of Planets in Binary Systems,'' {\em Astronomical Journal} 117, 
available from \url{http://citeseer.ist.psu.edu/358720.html}.

Read this paper and then modify your planetary simulation to
replicate or extend the results.
\end{ex}
